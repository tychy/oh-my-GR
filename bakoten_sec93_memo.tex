\documentclass{jsarticle}
\usepackage{amssymb,amsmath}
\usepackage{newtxtt}
\usepackage[utf8]{inputenc}
\newcommand{\pder}[2][]{\frac{\partial#1}{\partial#2}}
\newcommand{\half}{\frac{1}{2}}
\begin{document}
\section{リーマン曲率テンソルの定義}
\subsection{曲線座標のストークスの定理}
\subsection{リーマンテンソルの対称性}
\subsection{ビアンキの恒等式}
\subsection{Ricchiテンソル}
\section{スカラー曲率}
\subsection{2次元}
\subsection{3次元}
\subsection{4次元}
\subsection{ワイルテンソルについて}
\section{測地線方程式}
\subsection{特殊相対論における運動方程式}
\subsection{アフィンパラメーター}
\subsection{Hamilton-Jacobi}
\subsection{アイコナール方程式}
\subsection{測地線偏差の方程式}
\section{重力場の作用関数}
\subsection{電磁気の場合}
\subsection{重力場の作用関数}
\end{document}

