\documentclass{jsarticle}
\usepackage{amssymb,amsmath}
\usepackage{newtxtt}
\usepackage[utf8]{inputenc}
\usepackage{GR}
\newcommand{\pder}[2][]{\frac{\partial#1}{\partial#2}}
\newcommand{\dder}[2][]{\frac{\mathrm{d}#1}{\mathrm{d}#2}}
\newcommand{\half}{\frac{1}{2}}
\newcommand{\pri}[1]{#1^{\prime}}
\newcommand{\beq}{\begin{equation}}
\newcommand{\beql}[1]{\begin{equation}\label{#1}}
\newcommand{\eeq}{\end{equation}}
\newcommand{\eeqp}{\;\;\;.\end{equation}}
\newcommand{\eeqc}{\;\;\;,\end{equation}}
\begin{document}
\section{リーマン曲率テンソルの定義}
\subsection{曲線座標のストークスの定理}
\subsection{リーマンテンソルの対称性}
\subsection{ビアンキの恒等式}
\subsection{Ricchiテンソル}
\section{スカラー曲率}
\subsection{2次元}
\subsection{3次元}
\subsection{4次元}
\subsection{ワイルテンソルについて}
\section{測地線方程式}
\subsection{特殊相対論における運動方程式}
\subsection{アフィンパラメーター}
\subsection{Hamilton-Jacobi}
\subsection{アイコナール方程式}
\subsection{測地線偏差の方程式}
\section{重力場の作用関数}
\subsection{最小作用の原理}
粒子の運動は最小作用の原理に従うことをここで要請する。
つまり、作用$S$が、あってその作用を最小にするような経路を粒子が通ることを要請する。
ここで、作用は:
\beql{eq:S}
    S = \int_{a}^{b} - \alpha ds
\eeqp
と書く。$\alpha$は正の定数で、なぜなら
\beq \int_a^b ds\eeqp は、静止系で最大値を取るからである。
証明:
静止している時計$dt$に対して動いている時計$\pri{dt}$を考える。動いている系において動いている時計は静止している。
世界感覚の普遍性から、
\begin{align} 
    ds^2 = cdt^2 - dx^2 - dy^2 - dz^2 = cdt^{\prime 2}\\
    \pri{dt} = dt \sqrt{1 - \frac{v^2}{c^2}} 
\end{align}
つまり、$dt \geq \pri{dt}$がわかる。動いている時計は静止している時計よりも常に読みが小さい。
時計の読みは、$\frac{1}{c} \int ds$に等しい。静止している時計の世界線は真っ直ぐであるから、真っ直ぐな世界線に対する作用が最大であることがわかる。
\subsection{電磁気の場合}
重ね合わせの原理を要請する。
\subsection{重力場の作用関数}
電磁場のときと同じように、重力場の方程式はポテンシャルの二階導関数までで書けることを要請する。
測地線方程式を運動方程式と見ればクリストッフェル記号が力、計量がポテンシャルに当たる。
変分を取るときに次数が一つ増える(?)ので、ラグランジアン密度は計量の一階導関数までで書かれる。
しかし、計量とクリストッフェル記号から不変量を作り出すことはできないので二階導関数の線形な項も加えて不変量を作る。
ここで、変分を取る際に積分限界で計量と計量の一階導関数を固定すれば、二階導関数は表面積分で落ちる。
また、そのようなスカラー場として我々はすでにスカラー曲率を知っている。したがって、
\beq
    S_g = \int R \sqrt{-g} d\Omega
\eeq
\end{document}

