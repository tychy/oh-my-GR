\documentclass[twocolumn]{jsarticle}
\usepackage{amssymb,amsmath}
\usepackage{newtxtt}
\usepackage[utf8]{inputenc}
\usepackage{GR}
\newcommand{\pder}[2][]{\frac{\partial#1}{\partial#2}}
\newcommand{\dder}[2][]{\frac{\mathrm{d}#1}{\mathrm{d}#2}}
\newcommand{\ppder}[2][]{\frac{\partial^2#1}{{\partial#2}^2}}
\newcommand{\pikder}[3][]{\frac{\partial^2#1}{{\partial#2 \partial#3}}}
\newcommand{\pikdergx}[3][]{\frac{\partial^2 g_{#1}}{{\partial x^{#2} \partial x^{#3}}}}
\newcommand{\pderx}[2][]{\pder[#1]{x^{#2}}}
\newcommand{\pdergx}[2][]{\pderx[g_{#1}]{#2}}
\newcommand{\half}{\frac{1}{2}}
\newcommand{\hfpt}{\hspace{5pt}}
\newcommand{\ddfrac}[2]{\frac{{#1}^2}{{#2}^2}}
\newcommand{\beq}{\begin{equation}}
\newcommand{\beql}[1]{\begin{equation}\label{#1}}
\newcommand{\eeq}{\end{equation}}
\newcommand{\eeqp}{\;\;\;.\end{equation}}
\newcommand{\eeqc}{\;\;\;,\end{equation}}
\newcommand{\GaT}[3]{\Gamma^{#1}_{#2 #3}}
\newcommand{\pderGaTx}[4]{\pderx[\GaT{#1}{#2}{#3}]{#4}}
\newcommand{\Christfinside}[3]{\pdergx[#3 #1]{#2} + \pdergx[#2 #3]{#1} - \pdergx[#1 #2]{#3}}
\newcommand{\Christf}[4]{\Gamma^{#1}_{#2 #3}=\half g^{#1 #4}(\Christfinside{#2}{#3}{#4})}
\newcommand{\Ricchiinside}[2]{\pder[\Gamma^l_{#1 #2}]{x^l} - \pder[\Gamma^l_{#1 l}]{x^{#2}} 
    + \GaT{l}{#1}{#2}\GaT{m}{l}{m} - \GaT{m}{#1}{l}\GaT{l}{#2}{m}}
\newcommand{\ze}{(0)}
\date{\today}
\author{山田龍}
\title{bakoten107-110}
\begin{document}
\maketitle
\section{107}
弱い重力波を考える。
重力相互作用の伝播速度が有限であるなら、重力波の存在が考えられる。
真空中の弱い重力場について、計量を
\beq
g_{ik} = g^{\ze}_{ik} + h_{ik} 
\eeq
一次までの精度に置いて、
\beq
g^{ik} = g^{\ze ik} - h^{ik} 
\eeq
\beq
g = g^{\ze}(1 + h)
\eeq
微小変換においては、
\beq
h^\pr_{ik} = h_{ik} - \pderx[\xi_i]{k}- \pderx[\xi_k]{i}
\eeq
$h_{ik}$にゲージを入れる。
\beq
    \pderx[\psi^k_i]{k} = 0, \psi^k_i = h^k_i - \half \delta^k_i h
\eeq
曲率テンソルは、
\beq
R_{iklm} = \half(\pikder[h_{im}]{x^k}{x^l}+\pikder[h_{kl}]{x^i}{x^m}-
\pikder[h_{km}]{x^i}{x^l}-\pikder[h_{il}]{x^k}{x^m})
\eeq
リッチテンソルは、
\begin{align}
    R_{ik} &= g^{lm}R_{limk}\\
           &\sim g^{\ze lm}R_{limk}\\
           &= \half(-g^{\ze lm}\pikder[h_{ik}]{x^l}{x^m} + \pikder[h^l_{i}]{x^l}{x^k}+ \pikder[h^l_{k}]{x^i}{x^l}-\pikder[h]{x^i}{x^k}) 
\end{align}
であるから、ゲージを入れれば、
\beq
R_{ik} = \half(-g^{\ze lm}\pikder[h_{ik}]{x^l}{x^m})
\eeq
と線形化される。ダランベルシアンを使って書き換えて、
\beq
R_{ik} = \half \Box h_{ik}
\eeq
\beq
\Box = \Delta - \frac{1}{c^2} \ppder[]{t}
\eeq
真空中を考えているので、アインシュタイン方程式は
\beq
\Box h_{ik} = 0
\eeq
これは重力波が光速で伝播することを示す。
平面重力波が、$h_{23},h_{22}=-h_{33}$都によって決まること、
エネルギー運動量擬テンソルが4個の任意関数によって与えられるが、4は任意の自由な重力場におけるものであること。

todo
\section{108}
曲がった空間時間における重力波について考える。
非ガリレイ
\beq
g_{ik} = g^{\ze}_{ik} + h_{ik} 
\eeq

\section{109}
\section{110}
\end{document}

