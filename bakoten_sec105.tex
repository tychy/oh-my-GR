\documentclass[twocolumn]{jsarticle}
\usepackage{amssymb,amsmath}
\usepackage{newtxtt}
\usepackage[utf8]{inputenc}
\usepackage{GR}
\newcommand{\pder}[2][]{\frac{\partial#1}{\partial#2}}
\newcommand{\dder}[2][]{\frac{\mathrm{d}#1}{\mathrm{d}#2}}
\newcommand{\ppder}[2][]{\frac{\partial^2#1}{{\partial#2}^2}}
\newcommand{\pikder}[3][]{\frac{\partial^2#1}{{\partial#2 \partial#3}}}
\newcommand{\pikdergx}[3][]{\frac{\partial^2 g_{#1}}{{\partial x^{#2} \partial x^{#3}}}}
\newcommand{\pderx}[2][]{\pder[#1]{x^{#2}}}
\newcommand{\pderxx}[2][]{\pder[x^{#1}]{x^{#2}}}
\newcommand{\pdergx}[2][]{\pderx[g_{#1}]{#2}}
\newcommand{\pderugx}[2][]{\pderx[g^{#1}]{#2}}
\newcommand{\half}{\frac{1}{2}}
\newcommand{\hfpt}{\hspace{5pt}}
\newcommand{\ddfrac}[2]{\frac{{#1}^2}{{#2}^2}}
\newcommand{\beq}{\begin{equation}}
\newcommand{\beql}[1]{\begin{equation}\label{#1}}
\newcommand{\eeq}{\end{equation}}
\newcommand{\eeqp}{\;\;\;.\end{equation}}
\newcommand{\eeqc}{\;\;\;,\end{equation}}
\newcommand{\GaT}[3]{\Gamma^{#1}_{#2 #3}}
\newcommand{\pderGaTx}[4]{\pderx[\GaT{#1}{#2}{#3}]{#4}}
\newcommand{\Christfinside}[3]{\pdergx[#3 #1]{#2} + \pdergx[#2 #3]{#1} - \pdergx[#1 #2]{#3}}
\newcommand{\Christf}[4]{\Gamma^{#1}_{#2 #3}=\half g^{#1 #4}(\Christfinside{#2}{#3}{#4})}
\newcommand{\Ricchiinside}[2]{\pder[\Gamma^l_{#1 #2}]{x^l} - \pder[\Gamma^l_{#1 l}]{x^{#2}} 
    + \GaT{l}{#1}{#2}\GaT{m}{l}{m} - \GaT{m}{#1}{l}\GaT{l}{#2}{m}}
\date{\today}
\author{山田龍}
\title{105}
\begin{document}
\maketitle
\section{物体から離れた場所での重力場}
重力場から十分離れた場所では計量はガリレイ的な計量に対して小さな補正が付いた形にかけることを予想する。
\beq
g_{ik} = g^{(0)}_{ik} + h_{ik}
\eeq
という形に書いて、$h$のメトリックがガリレイ的なメトリックであることを要請する。
いま、反変計量テンソルは2次の精度で
\beq
    g_{ik}g^{kl} = \delta^l_i
\eeq
から定義されて、
\beq
g^{ik} = g^{(0)ik}(1 + h +\half h^2-\half h^i_k h^k_l)
\eeq
導出:

$h_{ik}$が小さいという要請をしたが、
\beq
    x^{\pr i} \rightarrow x^i + \xi^i
\eeq
というような任意の$\xi$の変換が入る余地があることに注意する。
座標変換に対するテンソルの変換は、
\beq
h^{\pr}_{ik} = h_{ik} - \pder[\xi_i]{x^k} - \pder[\xi_k]{x^i} 
\eeq
$\because$計量の変換、
\begin{align}
    g^{\pr ik}(x^{\pr l}) &= g^{lm}(x^l) \pderxx[\pr i]{l} \pderxx[\pr k]{m}\\
                          &= g^{lm}(x^l)(\delta^i_l + \pderx[\xi^i]{l})(\delta^k_m + \pderx[\xi^k]{m})\\
                          &\sim g^{ik}(x^l) + g^{im}\pderx[\xi^k]{m}+ g^{lk}\pderx[\xi^i]{l}
\end{align}
ここで、
\beq
g^{\pr ik}(x^{\pr l}) =  g^{\pr ik}(x^l + \xi^l) = g^{\pr ik}(x^l) + \pderugx[ik]{l}\xi^l
\eeq
から、
\beq
g^{\pr ik}(x^{\pr l}) = g^{ik}(x^l) - \pderugx[ik]{l}\xi^l + g^{im}\pderx[\xi^k]{m}+ g^{lk}\pderx[\xi^i]{l}
\eeq
この後ろの3つの項は
\beq
    \xi^{i;k} + \xi^{k;i}
\eeq
と書かれる。
したがって、
\beq
\delta g^{ik} = \xi^{i;k} + \xi^{k;i}
\eeq
\beq
    g^{\pr ik} = g^{ik} + \delta g^{ik}
\eeq
もしキリング方程式が満たされるなら、計量はその座標変換について形を変えない。
\section{近似}
原点から遠く離れた場所でのシュヴァルツシルト計量は
\beq
    ds^2 = ds^{0 2} - \frac{2km}{c^2r}(dr^2 + c^2dt^2)
\eeq
であったから、
\beq
dr = n_\al dx^\al
\eeq
\beq
dr^2 = n_\al n_\be dx^\al dx^\be
\eeq
とデカルト座標に直して
\beq
h^{(1)}_{00} = -\frac{r_g}{r},h^{(1)}_{\al \be} = -\frac{r_g}{r}n_\al n_\be,h^{(1)}_{0 \al} = 0
\eeq
これが1次近似。

\end{document}

