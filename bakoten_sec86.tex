\documentclass{jsarticle}
\usepackage{amssymb,amsmath}
\usepackage{newtxtt}
\usepackage[utf8]{inputenc}
\newcommand{\pder}[2][]{\frac{\partial#1}{\partial#2}}
\begin{document}

\section{86節 クリストッフェル記号と計量テンソル}

\subsection{計量テンソルの共変導関数}
%は、平行移動によって内積が変化しないことと等価で、
%クリストッフェル記号によって共変微分を一意的に与えたことから導出される。
計量テンソルの共変導関数が0であることを示す。\\
$DA^{k}$はベクトルであるので、%ここあやしい
\begin{equation}
    DA_{i} = g_{ik} DA^{k}\label{eq:dag}
\end{equation}
ここで、$A_{i} = g_{ik} A^{k}$を直接用いて、
\begin{align}
    DA_{i} &= D(g_{ik}A^{k}) \notag\\
           &= A^{k}Dg_{ik} + g_{ik} DA^{k}\label{eq:adggda}
\end{align}
式\eqref{eq:dag}と式\eqref{eq:adggda}を比べて、$A^{k}$は任意のベクトルなので
計量テンソル$g_{ik}$の共変微分について:
\begin{equation}
    Dg_{ik} = 0 \label{eq:Dg}
\end{equation}
また、$g_{ik}g^{ik}=g_{ik}g^{ki}=\delta^{i}_{i}=4$より、
\begin{align}
    0 &= D(g_{ik}g^{ik}) \\
      &= g_{ik}Dg^{ik} + g^{ik}Dg_{ik}
\end{align}
\eqref{eq:Dg}より、
\begin{equation}
    Dg^{ik} = 0
\end{equation}
共変導関数についても、
\begin{align}
    g_{ik;l} = 0 \label{eq:gikl}\\
    g^{ik}_{;l} = 0
\end{align}

\subsection{クリストッフェル記号の計量テンソルによる表示}
\eqref{eq:gikl}をクリストッフェル記号を使って展開する。
85節の共変テンソルの導関数から、
\begin{align}
    g_{ik;l} &= \frac{\partial{g_{ik}}}{\partial{x^{l}}} - 
    \Gamma^{m}_{il} g_{mk} - \Gamma^{m}_{kl}g_{im}\\
    &= 0\\
    \frac{\partial{g_{ik}}}{\partial{x^{l}}} &= 
    \Gamma^{m}_{il} g_{mk} + \Gamma^{m}_{kl}g_{im}\\
    &= \Gamma_{k,il} + \Gamma_{i,kl}
    \label{eq:gikxl}\\
    \frac{\partial{g_{kl}}}{\partial{x^{i}}} 
    &= \Gamma_{k,li} + \Gamma_{l,ki}
    \label{eq:gklxi}\\
    \frac{\partial{g_{li}}}{\partial{x^{k}}} 
    &= \Gamma_{l,ik} + \Gamma_{i,lk}
    \label{eq:glixk}
\end{align}
クリストッフェル記号の添字の先頭は、$(i, k) (k, l) (l, i)$であるので、
\eqref{eq:gikxl} + \eqref{eq:glixk} - \eqref{eq:gklxi}を計算して、
\begin{align}
    \Gamma_{i,kl} &=
    \frac{1}{2}
    \left(
    \frac{\partial{g_{ik}}}{\partial{x^{l}}}
     +
    \frac{\partial{g_{li}}}{\partial{x^{k}}}
     -
    \frac{\partial{g_{lk}}}{\partial{x^{i}}}
    \right)
    \label{eq:gamma1}\\
    \Gamma^{i}_{kl} &= g^{im} \Gamma_{m,kl}\\
                    &=
    \frac{1}{2}g^{im} 
    \left(
    \frac{\partial{g_{mk}}}{\partial{x^{l}}}
     +
    \frac{\partial{g_{lm}}}{\partial{x^{k}}}
     -
    \frac{\partial{g_{lk}}}{\partial{x^{m}}}
    \right)
    \label{eq:gamma2}
\end{align}
\subsection{縮約されたクリストッフェル記号}
例えば、この章のテンソルの発散や、リッチテンソルの対称性の部分で使われる。\\
\eqref{eq:gamma2}から、
\begin{equation}
    \Gamma^{i}_{ki} =
    \frac{1}{2}g^{im} 
    \left(
    \frac{\partial{g_{mk}}}{\partial{x^{i}}}
     +
    \frac{\partial{g_{im}}}{\partial{x^{k}}}
     -
    \frac{\partial{g_{ik}}}{\partial{x^{m}}}
    \right)
\end{equation}
右辺だけに注目して、$(i, m)$の入れ替えを考えると第1項と第3項は打ち消し合う。このとき、左辺に$i$があることは影響しない。
\begin{equation}
    \Gamma^{i}_{ki} =
    \frac{1}{2}g^{im} 
    \frac{\partial{g_{im}}}{\partial{x^{k}}}\label{eq:gammaiki}
\end{equation}
ここで、行列式の微分について考える。
正方行列$A=(a_{ij})$の逆行列を$A^{-1}=(b_{ij})$、余因子を$\tilde{A}=(\tilde{a}_{ij})=(\triangle_{ji})$と書く。
$A\tilde{A}=(detA)E$について、$A\tilde{A} = (c_{ij})$と書くと:
\begin{align}
    c_{ij} &= \sum_{k}a_{ik}\tilde{A}_{kj}\\
           &= \sum_{k}a_{ik}\triangle_{jk}\\
           &= \begin{vmatrix}
                a_{11} & \dots & a_{1n}\\
                \hdotsfor{3}\\
                a_{j-1,1} & \dots & a_{j-1,n}\\
                a_{i1} & \dots & a_{in}\\
                a_{j+1,1} & \dots & a_{j+1,n}\\
                \hdotsfor{3}\\
                a_{n1} & \dots & a_{nn}
            \end{vmatrix}\label{eq:vmat}\\
           &= (detA)\delta_{ij}\label{eq:cij}\\
    \therefore A\tilde{A}=(detA)E\\
    A^{-1} = \frac{\tilde{A}}{detA}
\end{align}
\eqref{eq:vmat}から\eqref{eq:cij}で、同じ行があると行列式が0になることを用いた。
\begin{align}
    |A| &= \sum_{j}a_{ij}\triangle_{ji}\\
        &= \sum_{j}a_{ij}\tilde{a}_{ij}\\
    \frac{\partial{|A|}}{\partial{a_{ij}}} &= \tilde{a}_{ji}\\
                                           &= |A|b_{ji}\\
    \frac{\partial{|A|}}{\partial{a_{ij}}}\frac{\partial{a_{ij}}}{\partial{x}} &= |A|b_{ji}\frac{\partial{a_{ij}}}{\partial{x}}\\
    \frac{\partial{|A|}}{\partial{x}} &= |A|b_{ji}\frac{\partial{a_{ij}}}{\partial{x}}
\end{align}
ここで$A$を計量テンソルにして、
\begin{equation}
    \frac{\partial{g}}{\partial{x}} = gg^{ij}\frac{\partial{g_{ij}}}{\partial{x}} 
\end{equation}
また、$g_{ik}g^{ik}=4$より、$g^{ik}dg_{ik}=-g_{ik}dg^{ik}$であることから、
\begin{equation}
    \frac{\partial{g}}{\partial{x}} = -gg_{ij}\frac{\partial{g^{ij}}}{\partial{x}} 
\end{equation}
\eqref{eq:gammaiki}は書き換えることができて、
\begin{align}
    \Gamma^{i}_{ki} &= \frac{1}{2g}\pder[g]{x^k}\\%\frac{\partial{g}}{\partial{x^{k}}}\\
                    &= \pder[\ln{\sqrt{-g}}]{x^k}
\end{align}
となる。
また、$g^{kl}\Gamma^{i}_{kl}$について、
\begin{align}
    g^{kl}\Gamma^{i}_{kl} &=
    \frac{1}{2}g^{kl}g^{im} 
    \left(
    \frac{\partial{g_{mk}}}{\partial{x^{l}}}
     +
    \frac{\partial{g_{lm}}}{\partial{x^{k}}}
     -
    \frac{\partial{g_{lk}}}{\partial{x^{m}}}
    \right)\\
    &=
    g^{kl}g^{im} 
    \left(
    \frac{\partial{g_{ml}}}{\partial{x^{k}}}
     -
    \frac{1}{2}\frac{\partial{g_{lk}}}{\partial{x^{m}}}
    \right)\\
    &= 
    g^{kl}g^{im} 
    \frac{\partial{g_{ml}}}{\partial{x^{k}}}
     -
    \frac{1}{2}
    g^{kl}g^{im} 
    \frac{\partial{g_{lk}}}{\partial{x^{m}}}\label{eq:gg1}\\
    &= 
    g^{kl}\left(- \pder[g^{im}]{x^{k}}g_{ml}\right)
    +
    \frac{1}{2}g^{im}g_{kl}\pder[g^{kl}]{x^{m}}\label{eq:gg2}\\
    &=
    - \delta^{k}_{m} \pder[g^{im}]{x^k} + \frac{1}{2}g^{ik}g_{ml}\pder[g^{ml}]{x^{k}} 
    \quad\left(k \leftrightarrow m\right)\\
    &=
    - \pder[g^{ik}]{x^k} - \frac{1}{2g}g^{ik}\pder[g]{x^{k}}\\
    &=
    - \frac{1}{\sqrt{-g}}\pder[\sqrt{-g}g^{ik}]{x^{k}}
\end{align}
\eqref{eq:gg1}から\eqref{eq:gg2}で、
\begin{equation}
    g_{ik}g^{kl} = \delta^l_i
\end{equation}
から、
\begin{equation}
    g_{ik}\pder[g^{kl}]{x} = - g^{kl}\pder[g_{ik}]{x}
\end{equation}
という関係を使った。
\subsection{曲線座標における発散}
曲線座標におけるベクトルの発散を考える。
\begin{align}
    A^{i}_{;i} 
    &= 
    \pder[A^{i}]{x^i} + \Gamma^i_{ki}A^k\\
    &=
    \pder[A^{i}]{x^i} + A^k\pder[\ln{\sqrt{-g}}]{x^k}\\
    &=
    \frac{1}{\sqrt{-g}}\frac{\partial \sqrt{-g}A^{i}}{\partial{x^{i}}}
    \label{eq:aidiv}
\end{align}
曲線座標におけるテンソルの発散は、\\
曲線座標におけるラプラシアンについて、スカラー$\phi$を考える。
\begin{equation}
\phi_{;i} = \pder[\phi]{x^i}
\end{equation}
添字を上げて、
\begin{equation}
    \phi^{;i} = g^{ik}\pder[\phi]{x^k}
\end{equation}
\eqref{eq:aidiv}を使って発散を取ると、
\begin{equation}
    \phi^{;i}_{;i} = 
    \frac{1}{\sqrt{-g}}
    \frac{\partial}{\partial{x^{i}}}
    \left(
        \sqrt{-g}g^{ik} \frac{\partial\phi}{x^{k}}
    \right)
\end{equation}
ガウスの定理
\ref{eq:Dg}
\end{document}
