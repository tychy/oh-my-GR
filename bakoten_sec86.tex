\documentclass{jsarticle}
\usepackage{newtxtt}
\begin{document}

\section{86 クリストッフェル記号と計量テンソル}

\subsection{導きたい結果}

計量テンソルの共変微分が0であること
\begin{equation}
    Dg_{ik} = 0 \label{eq:Dg}
\end{equation}
クリストッフェル記号の計量テンソルによる表示(86.3)
\begin{equation}
    \Gamma^{i}_{kl} =
    g^{im} 
    \left(
    \frac{\partial{g_{mk}}}{\partial{x^{l}}}
     +
    \frac{\partial{g_{lm}}}{\partial{x^{k}}}
     -
    \frac{\partial{g_{lk}}}{\partial{x^{m}}}
    \right)
\end{equation}
縮約されたクリストッフェル記号

\begin{equation}
    \Gamma^{i}_{ki} =
    \frac{\partial{ln\sqrt{-g}}}{\partial{x^{k}}}
\end{equation}
曲線座標におけるベクトルの発散
\begin{equation}
    A^{i}_{;i} = 
    \frac{1}{\sqrt{-g}}\frac{\partial \sqrt{-g}A^{i}}{\partial{x^{i}}}
\end{equation}
曲線座標におけるテンソルの発散


曲線座標におけるラプラシアン
\begin{equation}
    \phi^{;i}_{;i} = 
    \frac{1}{\sqrt{-g}}
    \frac{\partial}{\partial{x^{i}}}
    \left(
        \sqrt{-g}g^{ik} \frac{\partial\phi}{x^{k}}
    \right)
\end{equation}
ガウスの定理

\subsection{計量テンソルの共変微分(86.1)}

\end{document}
