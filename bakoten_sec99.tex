\documentclass{jsarticle}
\usepackage{amssymb,amsmath}
\usepackage{newtxtt}
\usepackage{GR}
\usepackage[utf8]{inputenc}
\newcommand{\pder}[2][]{\frac{\partial#1}{\partial#2}}
\newcommand{\dder}[2][]{\frac{\mathrm{d}#1}{\mathrm{d}#2}}
\newcommand{\ppder}[2][]{\frac{\partial^2#1}{{\partial#2}^2}}
\newcommand{\pikder}[3][]{\frac{\partial^2#1}{{\partial#2 \partial#3}}}
\newcommand{\pikdergx}[3][]{\frac{\partial^2 g_{#1}}{{\partial x^{#2} \partial x^{#3}}}}
\newcommand{\pderx}[2][]{\pder[#1]{x^{#2}}}
\newcommand{\pdergx}[2][]{\pderx[g_{#1}]{#2}}
\newcommand{\half}{\frac{1}{2}}
\newcommand{\hfpt}{\hspace{5pt}}
\newcommand{\ddfrac}[2]{\frac{{#1}^2}{{#2}^2}}
\newcommand{\beq}{\begin{equation}}
\newcommand{\beql}[1]{\begin{equation}\label{#1}}
\newcommand{\eeq}{\end{equation}}
\newcommand{\eeqp}{\;\;\;.\end{equation}}
\newcommand{\eeqc}{\;\;\;,\end{equation}}
\newcommand{\GaT}[3]{\Gamma^{#1}_{#2 #3}}
\newcommand{\pderGaTx}[4]{\pderx[\GaT{#1}{#2}{#3}]{#4}}
\newcommand{\Christfinside}[3]{\pdergx[#1 #3]{#2} + \pdergx[#3 #2]{#1} - \pdergx[#1 #2]{#3}}
\newcommand{\Christf}[4]{\Gamma^{#1}_{#2 #3}=\half g^{#1 #4}(\Christfinside{#2}{#3}{#4})}
\newcommand{\Ricchiinside}[2]{\pder[\Gamma^l_{#1 #2}]{x^l} - \pder[\Gamma^l_{#1 l}]{x^{#2}} 
+ \GaT{l}{#1}{#2}\GaT{m}{l}{m} - \GaT{m}{#1}{l}\GaT{l}{#2}{m}}
\newcommand{\hnud}{\frac{\dot{\nu}}{2}}
\newcommand{\hlad}{\frac{\dot{\lambda}}{2}}
\newcommand{\hnup}{\frac{\nu^\prime}{2}}
\newcommand{\hlap}{\frac{\lambda^\prime}{2}}
\newcommand{\hnupp}{\frac{\nu^{\prime \prime}}{2}}
\newcommand{\hladd}{\frac{\ddot{\lambda}}{2}}
\newcommand{\nula}{\nu - \lambda}
\newcommand{\enula}{e^{\nula}}
\newcommand{\epikcf}{\frac{8 \pi k}{c^4}}
\newcommand{\nup}{\nu^\prime}
\newcommand{\lap}{\lambda^\prime}
\newcommand{\revrt}{\frac{1}{r^2}}
\newcommand{\rgr}{1 - \frac{r_g}{r}}

\date{\today}
\author{山田龍}
\title{99-100}
\begin{document}
\maketitle
\section{99ニュートンの法則}
\subsection{万有引力の法則}
$\S 87$で見たニュートン近似の下で計量テンソルは、
\beq
g_{00} = 1 + \frac{2\phi}{c^2},g_{0\al} = 0,g_{\al \be} = \delta_{\al \be}
\eeq
%ちゃんとかくtodo1
巨視的な運動も遅いと近似しているので4元速度は、$u^\alpha = 0, u^0 = u_0 = 1$となり、
巨視的物体のエネルギー運動量テンソルは、質量密度を$\mu = \sum_a m \delta(r-r_a)$と書いて、
\begin{align}
    T^k_{i} &= \mu c^2 u_i u^k\\
    T^0_0 &= \mu c^2\\
    T &= T^i_i = T^0_0 = \mu c^2
\end{align}
アインシュタイン方程式は、
\beq
    R^k_i = \frac{8\pi k}{c^4} (T^k_i - \half \delta^k_i T)
\eeq
であった。$R^0_0$は、
\begin{align}
    R^0_0 &= \frac{8\pi k}{c^4} (T^0_0 - \half T)\\
        &= \frac{4\pi k}{c^2} \mu
\end{align}
% i = k = \alphaのときがわからないtodo 2
$i \neq k$のとき、$R^k_i$は
\beq
    R^k_i = 0
\eeq
リッチテンソルを計算する。
まず、リッチテンソルの式(92.7)
\beq
    R_{ik} =  \pder[\Gamma^l_{ik}]{x^l} - \pder[\Gamma^l_{il}]{x^k}
    + \GaT{l}{i}{k}\GaT{m}{l}{m} - \GaT{m}{i}{l}\GaT{l}{k}{m}
\eeq
後ろの2つの項は、$\phi$についての二次の項なので、一次までの精度では落とせる。
\begin{align}
    R_{ik} &= \pder[\Gamma^l_{ik}]{x^l} - \pder[\Gamma^l_{il}]{x^k}\\
           %&= \half g^{lt} \left( \pderx[]{l} \left(\Christd{t}{i}{k}\right) - 
           %\pderx[]{k}\left(\Christd{t}{i}{t} \right)\right)\\
            &= \half g^{lt} 
            \left( \pderx[]{l}(\pdergx[kt]{i} + \pdergx[it]{k} - \pdergx[ik]{t}) - \pderx[]{k}(\pdergx[lt]{i})\right)\\
            &= \half g^{lt} \left(
            \pikdergx[kt]{l}{i} + \pikdergx[it]{l}{k} - \pikdergx[ik]{l}{t} - \pikdergx[lt]{k}{i}
            \right)\\
    R_{00} &= \half g^{lt} \left(
            \pikdergx[0t]{l}{0} + \pikdergx[0t]{l}{0} - \pikdergx[00]{l}{t} - \pikdergx[lt]{0}{0}
            \right)\\
           &\sim \half g^{\al \be}(- \pikdergx[00]{\al}{\be} )\\
           &= \half \ppder[g_{00}]{x^\al}\\
           &= \frac{1}{c^2} \ppder[\phi]{x^\al}
\end{align}
\beq
R^0_0 = \frac{1}{c^2} \ppder[\phi]{{x^\alpha}} = \frac{4\pi k}{c^2} \mu
\eeq
重力ポテンシャルは、ポアソン方程式をみたす。
\beq
    \Delta  \phi = 4 \pi k \mu
\eeq
この解は、
\beq
    \phi = -k \int \frac{\mu dV}{R}
\eeq
粒子が一つのときに、もう一つの粒子に働く力は、質量をそれぞれ$m,m^\prime$として
\begin{align}
    \phi &= - \frac{km}{R}\\
    F &= - m^\prime \pder[\phi]{R}\\
      &= -m \frac{kmm^\prime}{R^2}
\end{align}
ニュートンの万有引力の法則を得た。
質量密度を使ってポテンシャルエネルギーを、
\beq
U = \half \int \mu \phi dV
\eeq
と書く。
\subsection{ポテンシャルの多重極展開}
不変な重力場では、質量が正であるから双極子モーメントを$0$にする座標系を取ることができる。
したがって、すべての質量が位置$R_0$につくるポテンシャルは、$R_0$が質点の位置よりも十分大きいとして、
\beq
\phi = -k \int dV \left(\frac{\mu}{R_0} + \half \mu x_\al x_\be \pikder[]{X_\al}{X_\be}(\frac{1}{R_0}) + \cdots \right)
\eeq
ここで、
\beq
    \Delta \frac{1}{R_0} = \delta_{\al \be} x_\al x_\be \pikder[]{X_\al}{X_\be}(\frac{1}{R_0}) = 0
\eeq
から、$\phi^{(2)}$は$x_\al x_\be$の自由度を一つ減じて
\begin{align}
    \phi^{(2)} &= -k \int dV \mu (x_\al x_\be - \frac{1}{3}r^2 \delta_{\al \be})\pikder[]{X_\al}{X_\be}(\frac{1}{R_0})\\
               &= -k \int dV \frac{1}{6} D_{\al \be} \pikder[]{X_\al}{X_\be}(\frac{1}{R_0})
\end{align}
質量4重極モーメントテンソルとして、
\beq
D_{\al \be} = \int dV \mu(3x_\al x_\be - r^2 \delta_{\al \be})
\eeq
これは慣性モーメントテンソル
\beq
    J_{\al \be} = \int \mu (r^2 \delta_{\al \be} - x_\al x_\be)
\eeq
\beq
J_{\gamma \gamma} = \int 2 \mu r^2 dV 
\eeq
より、慣性モーメントテンソルで記述される。
\beq
D_{\al \be} = J_{\gamma \gamma}\delta_{\al \be} - 3 J_{\al \be}
\eeq
\subsection{一様な楕円体の作るポテンシャル}
一様な楕円体を考える。質量密度を$\mu$とする。楕円体の表面は、
%https://physics.stackexchange.com/questions/16412/the-gravitational-potential-of-ellipsoid
\beq
\ddfrac{x}{a} + \ddfrac{y}{b} + \ddfrac{z}{x} = 1
\eeq
楕円体の外部のポテンシャルは、
\beq
\phi = -\mu abck \pi \int_{\xi}^{\infty} \left(1 - \frac{x^2}{a^2+ s} - \frac{y^2}{b^2+ s}- \frac{z^2}{c^2+ s}\right)\frac{ds}{\Delta}
\eeq
内部のポテンシャルは、
\beq
\phi = -\mu abck \pi \int_{0}^{\infty} \left(1 - \frac{x^2}{a^2+ s} - \frac{y^2}{b^2+ s}- \frac{z^2}{c^2+ s}\right)\frac{ds}{\Delta}
\eeq

物体の重力エネルギーは、楕円体の全領域積分$U = \half \int \mu \phi dV$であったので計算すると、
\begin{align}
    U &= \half \int \mu \phi dV\\
      &= - \half \mu^2 abck \pi \int dV \int_{0}^{\infty} \left(1 - \frac{x^2}{a^2+ s} - \frac{y^2}{b^2+ s}- \frac{z^2}{c^2+ s}\right)\frac{ds}{\Delta}\\
      &= \half \mu^2 abck \pi \int dV \int_{0}^{\infty} \left(\frac{x^2}{a^2+ s} + \frac{y^2}{b^2+ s}+ \frac{z^2}{c^2+ s} - 1 \right)\frac{ds}{\Delta}\\
\end{align}
$x = a x^\prime, y = b y^\prime , z = c z^\prime$と変数変換して$x_i^\prime$を$x_i$に書き換える。球体の体積積分に書き換えられて、
\begin{align}
U &= \half \mu^2 a^2 b^2 c^2 k \pi \int \int \int r^2 \sin\theta dr d\theta d\phi \int_{0}^{\infty} \left(\frac{a^2 x^2}{a^2+ s} + \frac{b^2 y^2}{b^2+ s}+ \frac{c^2 z^2}{c^2+ s} - 1 \right)\frac{ds}{\Delta}\\
  &= \half \mu^2 a^2 b^2 c^2 k \pi \int \int \int \sin\theta dr d\theta d\phi \int_{0}^{\infty} \left(\frac{a^2 r^4 \sin^2\theta\cos^2\phi}{a^2+ s} + \frac{b^2 r^4 \sin^2\theta \sin^2 \phi}{b^2+ s}+ \frac{c^2 r^4 \cos^2\theta}{c^2+ s} - r^2 \right)\frac{ds}{\Delta}\\
  &= \half \mu^2 a^2 b^2 c^2 k \pi \int \int \sin\theta d\theta d\phi \int_{0}^{\infty} \left( \frac{1}{5}\left(\frac{a^2 \sin^2\theta\cos^2\phi}{a^2+ s} + \frac{b^2 \sin^2\theta \sin^2 \phi}{b^2+ s}+ \frac{c^2 \cos^2\theta}{c^2+ s}\right) - \frac{1}{3} \right)\frac{ds}{\Delta}\\
  &=  \frac{2}{3} \mu^2 a^2 b^2 c^2 k \pi^2 \int_{0}^{\infty} \left( \frac{1}{5} \left(\frac{a^2}{a^2+ s} + \frac{b^2}{b^2+ s}+ \frac{c^2}{c^2+ s} \right) - 1 \right)\frac{ds}{\Delta}\\
  &= \frac{3km^2}{8} \int_0^\infty \left[\frac{1}{5}\left(\frac{a^2(b^2 + s)(c^2 + s)+(a^2 + s)b^2(c^2 + s)+ (a^2 + s)(b^2 + s)c^2}{\Delta^2} - 3 \right) - \frac{2}{5}\right]\frac{ds}{\Delta}\\
  &= \frac{3km^2}{8} \int_0^\infty \left[\frac{1}{5}\left(-s\frac{(b^2 + s)(c^2 + s)+(a^2 + s)(c^2 + s)+ (a^2 + s)(b^2 + s)}{\Delta^2}\right) - \frac{2}{5}\right]\frac{ds}{\Delta}\\
  &= \frac{3km^2}{8} \int_0^\infty \left[\frac{1}{5}\left(-s\frac{2\Delta\Delta^\prime}{\Delta^2}\right) - \frac{2}{5}\right]\frac{ds}{\Delta}\\
  &= \frac{3km^2}{8} \int_0^\infty \left[\frac{2s}{5}\dder[]{s}( \frac{1}{\Delta}) - \frac{2}{5}\frac{1}{\Delta}\right]ds\\
  &= \frac{3km^2}{8} \int_0^\infty \left[-\frac{2}{5}\frac{1}{\Delta} - \frac{2}{5}\frac{1}{\Delta}\right]ds\\
  &= - \frac{3km^2}{10} \int_0^\infty \frac{1}{\Delta}ds
\end{align}
この楕円積分は、
$a=b>c$の扁平な回転楕円体では、
\beq
U = - \frac{3km}{5\sqrt{a^2-c^2}} \cos^{-1}\frac{c}{a}
\eeq
$a>b=c$の縦長の回転楕円体では、
\beq
U = - \frac{3km^2}{5 \sqrt{a^2 - c^2}}cosh^{-1} \frac{a}{c}
\eeq
$a=b=c$の球体のときは、
\begin{align}
    U &= - \frac{3km^2}{10} \int_0^\infty \frac{1}{\Delta}ds\\
      &= - \frac{3km^2}{10} \int_0^\infty (a^2 + s)^{-\frac{3}{2}}ds\\
      &= - \frac{3km^2}{5a} 
\end{align}
%todo 3
%\subsection{問題}
%全体として等加速度で回転している、重力を及ぼすことのできる一様な液体の塊の釣り合いの形を求めよ\\
%解答:\\
%todo 4
\section{100}
球対称な系を考える。
まず、ガリレイ的な空間では
\beq
ds^2 = cdt^2 - dr^2 - r^2(d\theta^2 + \sin^2\theta d\phi^2)
\eeq
球面上の曲線の長さから立体角を定義する。$dt = dr = 0$を考えて、
\beq
    dl^2 = r^2 (d\theta^2 + \sin^2\theta d\phi^2) \equiv r^2 d\Omega^2
\eeq
中心対称な重力場での二次元球について考える。場が中心対称ということは、線要素が
\beq
    dl^2 = f(r, t)d\Omega^2
\eeq
のように書かれ、このとき球面の面積は$4\pi f(r,t)$である。
動径ベクトルを$f(r, t) \equiv r^{\prime 2}$と書き換えると、球面の面積は$4\pi r^{\prime 2}$になる。
いま、$(r,t)$から$(r^\prime , t^\prime)$への変換をした。この変換では球対称性は失わない。ここで、$(r^\prime, t^\prime)$を$(r,t)$に書き換える。(空間は、rの割り付けられた二次元球面をt=一定の三次元空間に張り巡らせたもの。各二次元球面の中心はその二次元球面上にはない。)\\
世界間隔は、
\beq
ds^2 = l(r, t) dt^ 2 -r^2 d\Omega^2 + h(r, t) dr^2 + a(r, t) dr dt
\eeq
$d\phi, d\theta$の一次の項は、$\phi \rightarrow -\phi$、$\theta \rightarrow \pi - \theta$の角度反転の対称性から消えた。
$a(r, t)$は静的な重力場であることを要請して、$t \rightarrow -t$で計量が不変であることを考えると消える。そうでなくても、$(r^\prime, t^\prime)$の変換の任意性から消すことができる。(キーワード:積分因子?)
%積分因子について書くb todo3
\begin{align}
    ds^2 &= l(r, t) dt^ 2 -r^2 d\Omega^2 + h(r, t) dr^2\\
         &= e^\nu c^2 dt^ 2 -r^2 (d\theta^2 + \sin^2\theta d\phi^2) - e^\lambda dr^2
\end{align}
計量テンソルの$0$でない成分は、
\begin{align}
    g_{00} = e^\nu \hfpt , g_{11} = -e^\lambda\hfpt ,g_{22} = -r^2\hfpt ,g_{33} = -r^2 \sin^2\theta\\
    g^{00} = e^{-\nu} \hfpt , g^{11} = -e^{-\lambda\hfpt} ,g^{22} = -r^{-2}\hfpt ,g^{33} = -r^{-2} \sin^{-2}\theta\\
\end{align}
クリストッフェル記号の成分を計算すると、$\Gamma^i_{kl}=\half g^{it}(\pdergx[kt]{l} + \pdergx[tl]{k} - \pdergx[kl]{t})$から、$0$でない項は、
\begin{align}
    \Christf{0}{0}{0}{t} &=  \half g^{00} (\Christfinside{0}{0}{0}) = \frac{\dot{\nu}}{2}\\
    \Christf{1}{1}{1}{t} &=  \half g^{11} (\Christfinside{1}{1}{1}) = \frac{\lambda^\prime}{2}\\
    \Christf{0}{1}{0}{t} &=  \half g^{00} (\Christfinside{1}{0}{0}) = \frac{\nu^\prime}{2}\\
    \Christf{0}{1}{1}{t} &=  \half g^{00} (\Christfinside{1}{1}{0}) = \frac{\dot{\lambda}}{2}e^{\lambda - \nu}\\
    \Christf{1}{1}{0}{t} &=  \half g^{11} (\Christfinside{1}{0}{1}) = \frac{\dot{\lambda}}{2}\\
    \Christf{1}{0}{0}{t} &=  \half g^{11} (\Christfinside{0}{0}{1}) = \frac{\nu^\prime }{2}e^{\nu - \lambda}\\
    \Christf{1}{2}{2}{t} &=  \half g^{11} (\Christfinside{2}{2}{1}) = -re^{-\lambda}\\
    \Christf{1}{3}{3}{t} &=  \half g^{11} (\Christfinside{3}{3}{1}) = -r\sin^2 \theta e^{-\lambda}\\
    \Christf{2}{3}{3}{t} &=  \half g^{22} (\Christfinside{3}{3}{2}) = -\sin\theta \cos\theta\\
    \Christf{3}{2}{3}{t} &=  \half g^{33} (\Christfinside{2}{3}{3}) = -\cot\theta\\
    \Christf{2}{1}{2}{t} &=  \half g^{22} (\Christfinside{1}{2}{2}) = \frac{1}{r}\\
    \Christf{3}{1}{3}{t} &=  \half g^{33} (\Christfinside{1}{3}{3}) = \frac{1}{r}\\
\end{align}
リッチテンソルの成分を計算すると、
\beq
    R_{ik} =  \pder[\Gamma^l_{ik}]{x^l} - \pder[\Gamma^l_{il}]{x^k}
    + \GaT{l}{i}{k}\GaT{m}{l}{m} - \GaT{m}{i}{l}\GaT{l}{k}{m}
\eeq
\begin{align}
    R_{00} &= \Ricchiinside{0}{0}\\
           %&= \pderGaTx{l}{0}{0}{l} + \pderGaTx{l}{0}{l}{0} + \GaT{l}{0}{0}\GaT{m}{l}{m}\\
    \pderGaTx{l}{0}{0}{l} &= \pderGaTx{0}{0}{0}{0} + \pderGaTx{1}{0}{0}{1}\\
                          &= \pderx[\frac{\dot{\nu}}{2}]{0} + \pderx[\frac{\nu^\prime }{2}e^{\nu - \lambda}]{1} \\
                          &= \pderx[\frac{\dot{\nu}}{2}]{0} + \frac{\nu^\prime }{2}\pderx[e^{\nu - \lambda}]{1} + e^{\nu - \lambda}\pderx[\frac{\nu^\prime }{2}]{1}\\ 
                          &= \pderx[\frac{\dot{\nu}}{2}]{0} + \frac{\nu^\prime }{2}(\nu^\prime - \lambda^\prime)e^{\nu - \lambda}
                          + e^{\nu - \lambda}\frac{\nu^{\prime \prime}}{2}\\ 
    \pderGaTx{l}{0}{l}{0} &= \pderGaTx{0}{0}{0}{0} + \pderGaTx{1}{0}{1}{0}\\
                          &= \pderx[\frac{\dot{\nu}}{2}]{0} + \pderx[\frac{\dot{\lambda}}{2}]{0}\\
    \GaT{l}{0}{0} \GaT{m}{l}{m} &= \GaT{0}{0}{0} \GaT{0}{0}{0} + \GaT{0}{0}{0} \GaT{0}{1}{1} + \GaT{1}{0}{0} \GaT{0}{1}{0} + \GaT{1}{0}{0} \GaT{0}{1}{0} + \GaT{1}{0}{0} \GaT{2}{1}{2} + \GaT{1}{0}{0} \GaT{3}{1}{3}\\
                                &= (\hnud)^2 + \hnud \hlad + \hnup e^{\nu - \lambda} \hnup + \hnup e^{\nu - \lambda} \hlad + \hnup e^{\nu - \lambda} \frac{1}{r} +\hnup e^{\nu - \lambda} \frac{1}{r}\\
                                \GaT{m}{i}{l} \GaT{l}{k}{m} &= \GaT{0}{0}{0} \GaT{0}{0}{0} + \GaT{1}{0}{0} \GaT{0}{0}{1} + \GaT{0}{0}{1} \GaT{1}{0}{0}\\
                                                            &= (\hnud)^2 + \hnup e^{\nu - \lambda} \hnup + \hnup \hnup e^{\nu - \lambda} + (\hlad)^2\\
                                R_{00} &= \hnupp \enula  + \hnup \enula (\nu^\prime - \lambda^\prime) - \hladd + \hnud \hlad + \hnup \enula \hlap + \hnup \enula \frac{2}{r} - (\hnup)^2 \enula - (\hlad)^2
\end{align}
同じ手順を繰り返せば、
\begin{align}
    R_{11} &= \hladd \enula + \hlad \enula (\dot{\lambda} - \dot{\nu}) - \hnupp + \frac{2}{r^2} + \hlad \enula \hnud + \hlap \hnup + \frac{\lap}{r} - (\hnup)^2 - \hlad \enula \hlad - \frac{2}{r^2}\\ 
    R_{22} &= -e^{-\lambda} + re^{-\lambda}\lap + \frac{1}{\sin^2\theta} - re^{-\lambda}\hnup - re^{-\lambda}\hlap - \cot^2 \theta\\
    R_{33} &= \sin^2\theta R_{22}\\
    R_{01} &= \frac{\dot{\lambda}}{r}
\end{align}
スカラー曲率は、
\begin{align}
    R &= g^{00} R_{00} + g^{11} R_{11}+ g^{22} R_{22}+ g^{22} R_{33}\\
      &= - \lambda^{\prime \prime} + \nu^{\prime \prime} + 2 \frac{\nu^\prime}{r} e^{-\lambda} - \frac{2}{r^2} - \frac{2\lambda^\prime}{r}e^{-\lambda} + \hnup e^{-\lambda} (\nu^\prime - \lambda^\prime) - \hlad e^{-\nu}(\dot{\lambda} - \dot{\nu}) + \frac{2}{r^2} e^{-\lambda}\\
\end{align}
重力場の方程式は、
\beq
R_{ik} - \half g_{ik} R = \frac{8 \pi k}{c^4}T_{ik}
\eeq
を計算して、
\begin{align}
    \epikcf T^0_0 &= R^0_0 - \frac{R}{2} = (\frac{\lambda^\prime}{r} - \frac{1}{r^2})e^{-\lambda} + \frac{1}{r^2} \label{eq:t00}\\ 
    \epikcf T^1_1 &= - (\frac{\nu^\prime}{r} + \frac{1}{r^2})e^{-\lambda} + \revrt \label{eq:t11}\\
    \epikcf T^1_0 &= - e^{-\lambda} \frac{\dot{\lambda}}{r}
\end{align}
この方程式は、質量の外側の中心対称な場を考えると質量の外側ではエネルギー運動量テンソルが$0$になることから、
\begin{align}
(\frac{\lambda^\prime}{r} - \frac{1}{r^2})e^{-\lambda} + \frac{1}{r^2} = 0\\ 
-(\frac{\nu^\prime}{r} + \frac{1}{r^2})e^{-\lambda} + \revrt = 0\label{eq:10009}\\
- e^{-\lambda} \frac{\dot{\lambda}}{r} = 0 \label{eq:10010}
\end{align}
\eqref{eq:10010}は、
\beq
\dot{\lambda} = 0 
\eeq
と書き換えられ、$\lambda = \lambda (r)$と時間に依らない形に書ける。\eqref{eq:t00}\eqref{eq:t11}から、
\beq
\lap + \nup = 0
\eeq
記号の肩の$\prime$はrでの微分であったので、
\beq
\lambda + \nu = f(t)
\eeq
$t=f(t^\prime)$の変換を$\lambda + \nu = 0$が満たされるように行って、$\nu$もtによらない形なるので、真空中の中心対称な重力場は静的な場になる。
\eqref{eq:10009}を積分する。両辺に$r^2$をかけて、
\begin{align}
r\lap e^{-\lambda} - e^{-\lambda} + 1 = 0\\
1 = (re^{-\lambda})^\prime\\
re^{-\lambda} = r + C
\end{align}
\beq
e^{-\lambda} = e^{\nu} = 1 + \frac{const}{r}
\eeq
無限遠で、万有引力の法則が成り立っていることを要請すると、rが大きい場合に
\beq
g_{00} = 1 + \frac{2\phi}{c^2}
\eeq
となることを要請することと同じなので、
\beq
const =  - \frac{2km}{c^2}
\eeq
世界間隔は、
\beq
ds^2 = (1 - \frac{r_g}{r})c^2dt^2 - r^2(\sin^2 \theta d\phi^2 +d\theta^2) - \frac{dr^2}{1-\frac{r_g}{r}}
\eeq
$r_g = \frac{2km}{c^2}$はシュヴァルツシルト半径または重力半径という。
アインシュタイン方程式のこの解をシュヴァルツシルト解という。任意の中心対称な質量分布に対して、質量が静止していなくても、真空中の重量場を与える。
空間的計量は、
\beq
dl^2 = \frac{dr^2}{1-\frac{r_g}{r}} + r^2 (\sin^2d\phi^2 + d\theta^2)
\eeq
rは円周が$2 \pi r$円の面積は$4 \pi r^2$になるようとった座標である。
同じ半径上の、二次元円の表面に垂直に定義された動径ベクトルに沿った積分は、$r>r_g$で
\beq
\int^{r_2}_{r_1} \frac{dr}{\sqrt{1 - \frac{r_g}{r}}} > r_2 - r_1\label{eq:10017}
\eeq
また、$g_{00} = 1 - \frac{r_g}{r} \leq 1$がわかる。真の時間は、
\beq
d\tau = \sqrt{g_{00}} dt \leq dt
\eeq
等号成立は$r \rightarrow \infty$、これは無限遠の観測者から見ると重力場の中の時計が遅れることを意味する。
十分離れたところでは、
\begin{align}
    ds^2 &= (1 - \frac{r_g}{r})c^2dt^2 - r^2(\sin^2 \theta d\phi^2 +d\theta^2) - \frac{dr^2}{1-\frac{r_g}{r}}\label{eq:ds}\\
&= c^2 dt^2 - dr^2 - r^2 d\Omega^2 - \frac{r_g}{r} (c^2 dt^2 + dr^2)\\
&= ds_0^2 - \frac{r_g}{r} (c^2 dt^2 + dr^2)
\end{align}
$ds_0^2$はガリレイの計量を意味する。\\
質量の内部の様子について考える。$r \rightarrow 0$での、重力場の方程式
\begin{align}
\epikcf T^0_0 &= (\frac{\lambda^\prime}{r} - \frac{1}{r^2})e^{-\lambda} + \frac{1}{r^2} \label{eq:t00ag}\\ 
    \epikcf T^1_1 &= - (\frac{\nu^\prime}{r} + \frac{1}{r^2})e^{-\lambda} + \revrt \label{eq:t11ag}
\end{align}
を見ると、\eqref{eq:t00ag}から$T^0_0$が$r=0$で特異点を持たないためには、$\lambda$が0に向かうことがわかる。
\begin{align}
\epikcf T^0_0 r^2 &= (r \lambda^\prime - 1)e^{-\lambda} + 1\\ 
\epikcf \int^r_0  T^0_0 r^2 &= \int^r_0 (r \lambda^\prime - 1)e^{-\lambda} + 1\\ 
                            &= [-re^{-\lambda}]^r_0 + r = -r e^{-\lambda} + r = r(1 - e^{-\lambda})\\ 
\lambda &= - ln\left(1 - \epikcf \frac{1}{r} \int^r_0  T^0_0 \right)\label{eq:lambda}
\end{align}
$T^0_0$の符号は一般的には決まった符号ではないが、ここでは
\beq
T^0_0 = g^{0i}T_{0i} = g^{00} T_{00} = e^{-\nu} T_{00} \geq 0
\eeq
$\lambda$の式\eqref{eq:lambda}を見ると$1 - \epikcf \frac{1}{r} \int^r_0  T^0_0 \geq 1$より、
\beq
\lambda \geq 0
\eeq
\beq
e^\lambda \geq 1
\eeq
\eqref{eq:t00ag} - \eqref{eq:t11ag}は、巨視的な物体のエネルギー運動量テンソルの表式$T^k_i = (p + \epsilon) u^k u_i - p \delta^{k}_i$と、
\beq
\frac{e^{-\lambda}}{r}(\nu^\prime + \lambda^\prime) = 
\epikcf (T^0_0 - T^1_1) = \epikcf\frac{(\epsilon + p)(1 + \frac{v^2}{c^2})}{1-\frac{v^2}{c^2}} \geq 0
\eeq
%todo 詳しく
よって、$\nu^\prime + \lambda^\prime \geq 0$。無限遠で計量がガリレイ的になるから、全空間で$\nu + \lambda \leq 0$がわかる。
$\lambda \geq 0$から$\nu \leq 0, e^\nu \leq 1$。これは、質量の内部でも時間が遅れることを示している。
エネルギー運動量テンソルは真空中では$0$なので、$r>a$の点に対して、
\beq
\lambda = - ln\left(1 - \epikcf \frac{1}{r} \int^a_0  T^0_0 \right)
\eeq
一方で、\eqref{eq:ds}から
\beq
g_{11} = - e^\lambda = - \frac{1}{1 - \frac{r_g}{r}}
\eeq
\beq
\lambda = - ln(1 - \frac{r_g}{r}) = - ln(1 - \frac{2km}{c^2 r})
\eeq
これを等しいとして、物体の質量がエネルギー運動量テンソルで表される。
\beq
m = \frac{4\pi}{c^2} \int_0^a T^0_0 r^2 dr
\eeq
物体中の物質分布が静的であれば、$T^0_0 = \epsilon$なので、
\beq
m = \frac{4\pi}{c^2} \int_0^a \epsilon r^2 dr \label{eq:m}
\eeq
空間の動径方向の距離は$\sqrt{g_{11}}r$である。
空間の体積要素は、$dV = 4 \pi e^{\frac{\lambda}{2}} dr$であることと、$e^{\lambda} \geq 1$から
\eqref{eq:m}は実際の質量よりも小さい値になる。これは重力的質量欠損である。
%余裕があれば問題を解く
\section{101 中心対称な重力場の中での運動}
中心対称な重力場中の物体の運動を考える。
中心対称の場では運動は一つの平面上で起こる。その面を$\theta = \frac{\pi}{2}$にとる。\\
補足:
測地線方程式の$\theta$の式を考えて、
\begin{align}
    \frac{d\theta^2}{ds^2} + 2\GaT{2}{1}{2} u^1 u^2 + \GaT{2}{3}{3} u^3 u^3 &= 0\\
    \frac{d\theta^2}{ds^2} + \frac{2}{r} \frac{d\theta}{ds} \frac{dr}{ds} - \sin\theta \cos\theta (\frac{d\phi}{ds})^2 &= 0\\
    \dder[]{s} (r^2 \dder[\theta]{s}) = 0
\end{align}    
この式に対して、初期条件$\theta = \frac{\pi}{2}, \dder[\theta]{s} = 0$を考えることは中心対称    であることを壊さない。この初期条件で、 
\begin{align}  
     \dder[]{s} (r^2 \dder[\theta]{s}) = 0 
\end{align}    
を得る。$\theta$が一定な平面上で物体が運動することがわかる。 
光の場合も固有時ではないアフィンパラメーターを使って同じ議論ができる。\\
物体の軌道を決めるためにハミルトンヤコビ方程式を考える。
\beq
g^{ik} \pder[S]{x^i}\pder[S]{x^k} - mc^2 = 0
\eeq
計量は知っているので具体的に書き直して、
\beq\label{eq:hamiltonj}
(1 - \frac{r_g}{r})^{-1} (\frac{\partial S}{c \partial t})^2 - (1 - \frac{r_g}{r})(\pder[S]{r})^2 - \revrt (\pder[S]{\phi})^2 - m^2c^2 = 0
\eeq
($m$は運動する物体の質量、 $m^\prime$は中心物体の質量、$r_g = \frac{2 k m^\prime}{c^2}$)\\
ここでハミルトンヤコビ方程式に、$t,\phi$が陽に現れないから、2つは循環変数である。\\
\subsection{Hamilton-Jacobi方程式についての一般論}
(以下冗長かもしれないのでは必要に応じて)\\
ここではとくに、変数分離される変数が循環変数である場合を考えることに注意する。
変数分離してハミルトンヤコビを循環変数のみの式とそれ以外に分解する。
例えば、$t$について分離して書くと、
\beq
S(q, t) = W(q) + Y(t)
\eeq
と分離して、一般的なハミルトニアンが時間$t$に陽によらないハミルトンヤコビ方程式
\beq
    H(q,\pder[S]{q}) + \pder[S]{t} = 0
\eeq
に代入して
\begin{align}
    H(q, \pder[W]{q}) &= - \pder[Y]{t} = const =\alpha_1\\
    \left\{
\begin{array}{l}
    \pder[Y]{t} = -\alpha_1\\
    H(q, \pder[W]{q}) = \alpha_1\label{eq:tokusei}
\end{array}
\right.
\end{align}
を得る。任意定数は循環座標に対する保存量になる。\eqref{eq:tokusei}はHamiltonの特性関数$W(q,\alpha)$に対するハミルトンヤコビ方程式になっている。
つまり、\eqref{eq:tokusei}を解けば、$W(q,\alpha)$が得られる。ここで$al = {\al_1, ... ,\al_N}$と書き直したのは、変数$q$の数の分だけ積分定数が出てくることからである。
主関数$S$は、
\beq
S(q, \alpha, t) = W(q, \al) - \al_1 t
\eeq
とかける。主関数を母関数として正準変換した変数$\al, \be$についてのHamiltonの運動方程式を解けば、
ハミルトンヤコビ方程式の解から主関数を構成したので、$\al = const, \be = const$である。また、母関数$S$についての正準変換の関係式から、
\beq
 \be_i = \pder[S]{\al_i}
\eeq
(ここまで)
\subsection{}
ハミルトンヤコビ方程式\eqref{eq:hamiltonj}を解く。循環座標に注意して、解を
\beq
    S = - Et + M \phi + S_r(r)
\eeq
の形に置く。これを\eqref{eq:hamiltonj}に代入して、
\beq
    (1 - \frac{r_g}{r})(\pder[S]{r})^2 = (1 - \frac{r_g}{r})^{-1} \frac{E^2}{c^2} - (\frac{M^2}{r^2} + m^2c^2)
\eeq
よって、
\beq
S_r = \int \left[(1 - \frac{r_g}{r})^{-2} \frac{E^2}{c^2} - (\rgr)^{-1}(\frac{M^2}{r^2} + m^2c^2)
 \right]^{\half}dr
\eeq
ハミルトンヤコビ方程式の解の任意定数$\al_i$のそれぞれについて、(今のように不完全解においても)、
\beq
    \pder[S]{\al_i} = const
\eeq
は、$q_1, ... q_N$と$t$についての方程式を一つ与える。特に、$\al_i$が循環変数に対応した保存量であるならば、任意定数の付いていない項と$\al_i$を保存量とする循環変数との方程式が与えられる。具体的には、$\pder[S]{E} = const$では$r, t$、$\pder[S]{\phi} = const$では$r, \phi$についての方程式を得る。
\begin{align}
    const = -t + \pder[S]{E}\\
    t = -const + \int \frac{\frac{2E}{c^2} (\rgr)^{-2}}{2\left[(\rgr)^{-2} \frac{E^2}{c^2} - (\rgr)^{-1}(\frac{M^2}{r^2} + m^2c^2) \right]^{\half}} dr\\
    t = \int \frac{\frac{E}{c^2} }{(\rgr)\left[ \frac{E^2}{c^2} - (\rgr)(\frac{M^2}{r^2} + m^2c^2) \right]^{\half}} dr\\
    ct = \int \frac{\frac{E}{mc^2} }{(\rgr)\left[ \frac{E^2}{m^2c^4} - (\rgr)(1 + \frac{M^2}{m^2c^2r^2}) \right]^{\half}} dr
\end{align}
軌道は、
\begin{align}
    \phi + \pder[S_r]{M} = const\label{eq:kidou}\\
    \phi &= \int \frac{\frac{M}{r^2} (\rgr)^{-1}}{\left[(\rgr)^{-2} \frac{E^2}{c^2} - (\rgr)^{-1}(\frac{M^2}{r^2} + m^2c^2) \right]^{\half}} d\phi\\
&= \int \frac{\frac{M}{r^2}}{\left[\frac{E^2}{c^2} - (\rgr)(\frac{M^2}{r^2} + m^2c^2) \right]^{\half}} d\phi\\
\end{align}
太陽の引力の場の中の遊星の運動について考える。遊星の速度が高速に比べて小さいことは、$\frac{r_g}{r}$が小さいことを意味する。遊星の軌道に対する相対論補正を調べる。
動径方向

角度$\phi$

\subsection{光線の径路}
アイコナール方程式が光線の径路を決める。













\end{document}

