\documentclass{jsarticle}
\usepackage{amssymb,amsmath}
\usepackage{newtxtt}
\usepackage{GR}
\usepackage[utf8]{inputenc}
\newcommand{\pder}[2][]{\frac{\partial#1}{\partial#2}}
\newcommand{\dder}[2][]{\frac{\mathrm{d}#1}{\mathrm{d}#2}}
\newcommand{\ppder}[2][]{\frac{\partial^2#1}{{\partial#2}^2}}
\newcommand{\pikder}[3][]{\frac{\partial^2#1}{{\partial#2 \partial#3}}}
\newcommand{\pikdergx}[3][]{\frac{\partial^2 g_{#1}}{{\partial x^{#2} \partial x^{#3}}}}
\newcommand{\pderx}[2][]{\pder[#1]{x^{#2}}}
\newcommand{\pdergx}[2][]{\pderx[g_{#1}]{#2}}
\newcommand{\half}{\frac{1}{2}}
\newcommand{\hfpt}{\hspace{5pt}}
\newcommand{\ddfrac}[2]{\frac{{#1}^2}{{#2}^2}}
\newcommand{\beq}{\begin{equation}}
\newcommand{\beql}[1]{\begin{equation}\label{#1}}
\newcommand{\eeq}{\end{equation}}
\newcommand{\eeqp}{\;\;\;.\end{equation}}
\newcommand{\eeqc}{\;\;\;,\end{equation}}
\newcommand{\GaT}[3]{\Gamma^{#1}_{#2 #3}}
\newcommand{\Christfinside}[3]{\pdergx[#1 #3]{#2} + \pdergx[#3 #2]{#1} - \pdergx[#1 #2]{#3}}
\newcommand{\Christf}[4]{\Gamma^{#1}_{#2 #3}=\half g^{#1 #4}(\Christfinside{#2}{#3}{#4})}
\date{\today}
\author{山田龍}
\title{99-101}
\begin{document}
\maketitle
\section{99ニュートンの法則}
\subsection{万有引力の法則}
$\S 87$で見たニュートン近似の下で計量テンソルは、
\beq
g_{00} = 1 + \frac{2\phi}{c^2},g_{0\al} = 0,g_{\al \be} = \delta_{\al \be}
\eeq
%ちゃんとかくtodo1
巨視的な運動も遅いと近似しているので4元速度は、$u^\alpha = 0, u^0 = u_0 = 1$となり、
巨視的物体のエネルギー運動量テンソルは、質量密度を$\mu = \sum_a m \delta(r-r_a)$と書いて、
\begin{align}
    T^k_{i} &= \mu c^2 u_i u^k\\
    T^0_0 &= \mu c^2\\
    T &= T^i_i = T^0_0 = \mu c^2
\end{align}
アインシュタイン方程式は、
\beq
    R^k_i = \frac{8\pi k}{c^4} (T^k_i - \half \delta^k_i T)
\eeq
であった。$R^0_0$は、
\begin{align}
    R^0_0 &= \frac{8\pi k}{c^4} (T^0_0 - \half T)\\
        &= \frac{4\pi k}{c^2} \mu
\end{align}
% i = k = \alphaのときがわからないtodo 2
$i \neq k$のとき、$R^k_i$は
\beq
    R^k_i = 0
\eeq
リッチテンソルを計算する。
まず、リッチテンソルの式(92.7)
\beq
    R_{ik} =  \pder[\Gamma^l_{ik}]{x^l} - \pder[\Gamma^l_{il}]{x^k}
    + \GaT{l}{i}{k}|GaT{m}{l}{m} - \GaT{m}{i}{l}\GaT{l}{k}{m}
\eeq
後ろの2つの項は、$\phi$についての二次の項なので、一次までの精度では落とせる。
\begin{align}
    R_{ik} &= \pder[\Gamma^l_{ik}]{x^l} - \pder[\Gamma^l_{il}]{x^k}\\
           %&= \half g^{lt} \left( \pderx[]{l} \left(\Christd{t}{i}{k}\right) - 
           %\pderx[]{k}\left(\Christd{t}{i}{t} \right)\right)\\
            &= \half g^{lt} 
            \left( \pderx[]{l}(\pdergx[kt]{i} + \pdergx[it]{k} - \pdergx[ik]{t}) - \pderx[]{k}(\pdergx[lt]{i})\right)\\
            &= \half g^{lt} \left(
            \pikdergx[kt]{l}{i} + \pikdergx[it]{l}{k} - \pikdergx[ik]{l}{t} - \pikdergx[lt]{k}{i}
            \right)\\
    R_{00} &= \half g^{lt} \left(
            \pikdergx[0t]{l}{0} + \pikdergx[0t]{l}{0} - \pikdergx[00]{l}{t} - \pikdergx[lt]{0}{0}
            \right)\\
           &\sim \half g^{\al \be}(- \pikdergx[00]{\al}{\be} )\\
           &= \half \ppder[g_{00}]{x^\al}\\
           &= \frac{1}{c^2} \ppder[\phi]{x^\al}
\end{align}
\beq
R^0_0 = \frac{1}{c^2} \ppder[\phi]{{x^\alpha}} = \frac{4\pi k}{c^2} \mu
\eeq
重力ポテンシャルは、ポアソン方程式をみたす。
\beq
    \Delta  \phi = 4 \pi k \mu
\eeq
この解は、
\beq
    \phi = -k \int \frac{\mu dV}{R}
\eeq
粒子が一つのときに、もう一つの粒子に働く力は、質量をそれぞれ$m,m^\prime$として
\begin{align}
    \phi &= - \frac{km}{R}\\
    F &= - m^\prime \pder[\phi]{R}\\
      &= -m \frac{kmm^\prime}{R^2}
\end{align}
ニュートンの万有引力の法則を得た。
質量密度を使ってポテンシャルエネルギーを、
\beq
U = \half \int \mu \phi dV
\eeq
と書く。
\subsection{ポテンシャルの多重極展開}
不変な重力場では、質量が正であるから双極子モーメントを$0$にする座標系を取ることができる。
したがって、すべての質量が位置$R_0$につくるポテンシャルは、$R_0$質点の位置よりも十分大きいとして、
\beq
\phi = -k \int dV \left(\frac{\mu}{R_0} + \half \mu x_\al x_\be \pikder[]{X_\al}{X_\be}(\frac{1}{R_0}) + \cdots \right)
\eeq
ここで、
\beq
    \Delta \frac{1}{R_0} = \delta_{\al \be} x_\al x_\be \pikder[]{X_\al}{X_\be}(\frac{1}{R_0}) = 0
\eeq
から、$\phi^{(2)}$は$x_\al x_\be$の自由度を一つ減じて
\begin{align}
    \phi^{(2)} &= -k \int dV \mu (x_\al x_\be - \frac{1}{3}r^2 \delta_{\al \be})\pikder[]{X_\al}{X_\be}(\frac{1}{R_0})\\
               &= -k \int dV \frac{1}{6} D_{\al \be} \pikder[]{X_\al}{X_\be}(\frac{1}{R_0})
\end{align}
質量4重極モーメントテンソルとして、
\beq
D_{\al \be} = \int dV \mu(3x_\al x_\be - r^2 \delta_{\al \be})
\eeq
これは慣性モーメントテンソル
\beq
    J_{\al \be} = \int \mu (r^2 \delta_{\al \be} - x_\al x_\be)
\eeq
\beq
J_{\gamma \gamma} = \int 2 \mu r^2 dV 
\eeq
より、慣性モーメントテンソルで記述される。
\beq
D_{\al \be} = J_{\gamma \gamma}\delta_{\al \be} - 3 J_{\al \be}
\eeq
\subsection{一様な楕円体の作るポテンシャル}
一様な楕円体を考える。質量密度を$\mu$とする。楕円体の表面は、
%https://physics.stackexchange.com/questions/16412/the-gravitational-potential-of-ellipsoid
\beq
\ddfrac{x}{a} + \ddfrac{y}{b} + \ddfrac{z}{x} = 1
\eeq
物体の重力エネルギーは、$U = \half \int \mu \phi dV$であったので計算すると、
%todo 3
\subsection{問題}
全体として等加速度で回転している、重力を及ぼすことのできる一様な液体の塊の釣り合いの形を求めよ\\
解答:\\
%todo 4
\section{100}
球対称な系を考える。
まず、ガリレイ的な空間では
\beq
ds^2 = cdt^2 - dr^2 - r^2(d\theta^2 + \sin^2\theta d\phi^2)
\eeq
球面上の曲線の長さから立体角を定義する。$dt = dr = 0$を考えて、
\beq
    dl^2 = r^2 (d\theta^2 + \sin^2\theta d\phi^2) \equiv r^2 d\Omega^2
\eeq
中心対称な重力場での二次元球について考える。場が中心対称ということは、線要素が
\beq
    dl^2 = f(r, t)d\Omega^2
\eeq
のように書かれ、このとき球面の面積は$4\pi f(r,t)$である。
動径ベクトルを$f(r, t) \equiv r^{\prime 2}$と書き換えると、球面の面積は$4\pi r^{\prime 2}$になる。
いま、$(r,t)$から$(r^\prime , t^\prime)$への変換をした。この変換では球対称性は失わない。ここで、$(r^\prime, t^\prime)$を$(r,t)$に書き換える。(空間は、rの割り付けられた二次元球面をt=一定の三次元空間に張り巡らせたもの。各二次元球面の中心はその二次元球面上にはない。)\\
世界間隔は、
\beq
ds^2 = l(r, t) dt^ 2 -r^2 d\Omega^2 + h(r, t) dr^2 + a(r, t) dr dt
\eeq
$d\phi, d\theta$の一次の項は、$\phi \rightarrow -\phi$、$\theta \rightarrow \pi - \theta$の角度反転の対称性から消えた。
$a(r, t)$は静的な重力場であることを要請して、$t \rightarrow -t$で計量が不変であることを考えると消える。そうでなくても、$(r^\prime, t^\prime)$の変換の任意性から消すことができる。(キーワード:積分因子?)
%積分因子について書くb todo3
\begin{align}
    ds^2 &= l(r, t) dt^ 2 -r^2 d\Omega^2 + h(r, t) dr^2\\
         &= e^\nu c^2 dt^ 2 -r^2 (d\theta^2 + \sin^2\theta d\phi^2) - e^\lambda dr^2
\end{align}
計量テンソルの$0$出ない成分は、
\begin{align}
    g_{00} = e^\nu \hfpt , g_{11} = -e^\lambda\hfpt ,g_{22} = -r^2\hfpt ,g_{33} = -r^2 \sin^2\theta\\
    g^{00} = e^{-\nu} \hfpt , g^{11} = -e^{-\lambda\hfpt} ,g^{22} = -r^{-2}\hfpt ,g^{33} = -r^{-2} \sin^{-2}\theta\\
\end{align}
クリストッフェル記号の成分を計算すると、$\Gamma^i_{kl}=\half g^{it}(\pdergx[kt]{l} + \pdergx[tl]{k} - \pdergx[kl]{t})$から、$0$でない項は、
\begin{align}
    \Christf{0}{0}{0}{t} &=  \half g^{00} (\Christfinside{0}{0}{0}) = \frac{\dot{\nu}}{2}\\
    \Christf{1}{1}{1}{t} &=  \half g^{11} (\Christfinside{1}{1}{1}) = \frac{\lambda^\prime}{2}\\
    \Christf{0}{1}{0}{t} &=  \half g^{00} (\Christfinside{1}{0}{0}) = \frac{\nu^\prime}{2}\\
    \Christf{0}{1}{1}{t} &=  \half g^{00} (\Christfinside{1}{1}{0}) = \frac{\dot{\lambda}}{2}e^{\lambda - \nu}\\
    \Christf{1}{1}{0}{t} &=  \half g^{11} (\Christfinside{1}{0}{1}) = \frac{\dot{\lambda}}{2}\\
    \Christf{1}{0}{0}{t} &=  \half g^{11} (\Christfinside{0}{0}{1}) = \frac{\nu^\prime }{2}e^{\nu - \lambda}\\
    \Christf{1}{2}{2}{t} &=  \half g^{11} (\Christfinside{2}{2}{1}) = -re^{-\lambda}\\
    \Christf{1}{3}{3}{t} &=  \half g^{11} (\Christfinside{3}{3}{1}) = -r\sin^2 \theta e^{-\lambda}\\
    \Christf{2}{3}{3}{t} &=  \half g^{22} (\Christfinside{3}{3}{2}) = -\sin\theta \cos\theta\\
    \Christf{3}{2}{3}{t} &=  \half g^{33} (\Christfinside{2}{3}{3}) = -\cot\theta\\
    \Christf{2}{1}{2}{t} &=  \half g^{22} (\Christfinside{1}{2}{2}) = \frac{1}{r}\\
    \Christf{3}{1}{3}{t} &=  \half g^{33} (\Christfinside{1}{3}{3}) = \frac{1}{r}\\
\end{align}
リッチテンソルの成分を計算すると、
\beq
    R_{ik} =  \pder[\Gamma^l_{ik}]{x^l} - \pder[\Gamma^l_{il}]{x^k}
    + \GaT{l}{i}{k}|GaT{m}{l}{m} - \GaT{m}{i}{l}\GaT{l}{k}{m}
\eeq

\section{101}
















































\end{document}

