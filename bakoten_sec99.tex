\documentclass{jsarticle}
\usepackage{amssymb,amsmath}
\usepackage{newtxtt}
\usepackage{GR}
\usepackage[utf8]{inputenc}
\newcommand{\pder}[2][]{\frac{\partial#1}{\partial#2}}
\newcommand{\dder}[2][]{\frac{\mathrm{d}#1}{\mathrm{d}#2}}
\newcommand{\ppder}[2][]{\frac{\partial^2#1}{{\partial#2}^2}}
\newcommand{\pikder}[3][]{\frac{\partial^2#1}{{\partial#2 \partial#3}}}
\newcommand{\pikdergx}[3][]{\frac{\partial^2 g_{#1}}{{\partial x^{#2} \partial x^{#3}}}}
\newcommand{\pderx}[2][]{\pder[#1]{x^{#2}}}
\newcommand{\pdergx}[2][]{\pderx[g_{#1}]{#2}}
\newcommand{\half}{\frac{1}{2}}
\newcommand{\beq}{\begin{equation}}
\newcommand{\beql}[1]{\begin{equation}\label{#1}}
\newcommand{\eeq}{\end{equation}}
\newcommand{\eeqp}{\;\;\;.\end{equation}}
\newcommand{\eeqc}{\;\;\;,\end{equation}}
\date{\today}
\author{山田龍}
\title{99-101}
\begin{document}
\maketitle
\section{99ニュートンの法則}
\subsection{万有引力の法則}
$\S 87$で見たニュートン近似の下で計量テンソルは、
\beq
g_{00} = 1 + \frac{2\phi}{c^2},g_{0\al} = 0,g_{\al \be} = \delta_{\al \be}
\eeq
%ちゃんとかくtodo1
巨視的な運動も遅いと近似しているので4元速度は、$u^\alpha = 0, u^0 = u_0 = 1$となり、
巨視的物体のエネルギー運動量テンソルは、質量密度を$\mu = \sum_a m \delta(r-r_a)$と書いて、
\begin{align}
    T^k_{i} &= \mu c^2 u_i u^k\\
    T^0_0 &= \mu c^2\\
    T &= T^i_i = T^0_0 = \mu c^2
\end{align}
アインシュタイン方程式は、
\beq
    R^k_i = \frac{8\pi k}{c^4} (T^k_i - \half \delta^k_i T)
\eeq
であった。$R^0_0$は、
\begin{align}
    R^0_0 &= \frac{8\pi k}{c^4} (T^0_0 - \half T)\\
        &= \frac{4\pi k}{c^2} \mu
\end{align}
% i = k = \alphaのときがわからないtodo 2
$i \neq k$のとき、$R^k_i$は
\beq
    R^k_i = 0
\eeq
リッチテンソルを計算する。
まず、リッチテンソルの式(92.7)
\beq
    R_{ik} =  \pder[\Gamma^l_{ik}]{x^l} - \pder[\Gamma^l_{il}]{x^k}
+ \Gamma \Gamma - \Gamma \Gamma
\eeq
後ろの2つの項は、$\phi$についての二次の項なので、一次までの精度では落とせる。
\begin{align}
    R_{ik} &= \pder[\Gamma^l_{ik}]{x^l} - \pder[\Gamma^l_{il}]{x^k}\\
           %&= \half g^{lt} \left( \pderx[]{l} \left(\Christd{t}{i}{k}\right) - 
           %\pderx[]{k}\left(\Christd{t}{i}{t} \right)\right)\\
            &= \half g^{lt} 
            \left( \pderx[]{l}(\pdergx[kt]{i} + \pdergx[it]{k} - \pdergx[ik]{t}) - \pderx[]{k}(\pdergx[lt]{i})\right)\\
            &= \half g^{lt} \left(
            \pikdergx[kt]{l}{i} + \pikdergx[it]{l}{k} - \pikdergx[ik]{l}{t} - \pikdergx[lt]{k}{i}
            \right)\\
    R_{00} &= \half g^{lt} \left(
            \pikdergx[0t]{l}{0} + \pikdergx[0t]{l}{0} - \pikdergx[00]{l}{t} - \pikdergx[lt]{0}{0}
            \right)\\
           &\sim \half g^{\al \be}(- \pikdergx[00]{\al}{\be} )\\
           &= \half \ppder[g_{00}]{x^\al}\\
           &= \frac{1}{c^2} \ppder[\phi]{x^\al}
\end{align}
\beq
R^0_0 = \frac{1}{c^2} \ppder[\phi]{{x^\alpha}} = \frac{4\pi k}{c^2} \mu
\eeq
重力ポテンシャルは、ポアソン方程式をみたす。
\beq
    \Delta  \phi = 4 \pi k \mu
\eeq
この解は、
\beq
    \phi = -k \int \frac{\mu dV}{R}
\eeq
粒子が一つのときに、もう一つの粒子に働く力は、質量をそれぞれ$m,m^\prime$として
\begin{align}
    \phi &= - \frac{km}{R}\\
    F &= - m^\prime \pder[\phi][R]\\
      &= -m \frac{kmm^\prime}{R^2}
\end{align}
ニュートンの万有引力の法則を得た。
質量密度を使ってポテンシャルエネルギーを、
\beq
U = \half \int \mu \phi dV
\eeq
と書く。
\subsection{ポテンシャルの多重極展開}
不変な重力場では、質量が正であるから双極子モーメントを$0$にする座標系を取ることができる。
したがって、すべての質量が位置$R_0$につくるポテンシャルは、$R_0$質点の位置よりも十分大きいとして、
\beq
\phi = -k \int dV \left(\frac{\mu}{R_0} + \half \mu x_\al x_\be \pikder[]{X_\al}{X_\be}(\frac{1}{R_0}) + \cdots \right)
\eeq
ここで、
\beq
    \Delta \frac{1}{R_0} = \delta_{\al \be} x_\al x_\be \pikder[]{X_\al}{X_\be}(\frac{1}{R_0}) = 0
\eeq
から、$\phi^{(2)}$は$x_\al x_\be$の自由度を一つ減じて
\begin{align}
    \phi^{(2)} &= -k \int dV \mu (x_\al x_\be - \frac{1}{3}r^2 \delta_{\al \be})\pikder[]{X_\al}{X_\be}(\frac{1}{R_0})\\
               &= -k \int dV \frac{1}{6} D_{\al \be} \pikder[]{X_\al}{X_\be}(\frac{1}{R_0})
\end{align}
質量4重極モーメントテンソルとして、
\beq
D_{\al \be} = \int dV \mu(3x_\al x_\be - r^2 \delta_{\al \be})
\eeq

\subsection{一様な楕円体の作るポテンシャル}

\section{100}
\section{101}
\end{document}

