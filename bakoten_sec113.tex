\documentclass{jsarticle}
\usepackage{amssymb,amsmath}
\usepackage{newtxtt}
\usepackage[utf8]{inputenc}
\newcommand{\pder}[2][]{\frac{\partial#1}{\partial#2}}
\newcommand{\dder}[2][]{\frac{\mathrm{d}#1}{\mathrm{d}#2}}
\newcommand{\ppder}[2][]{\frac{\partial^2#1}{{\partial#2}^2}}
\newcommand{\pikder}[3][]{\frac{\partial^2#1}{{\partial#2 \partial#3}}}
\newcommand{\pikdergx}[3][]{\frac{\partial^2 g_{#1}}{{\partial x^{#2} \partial x^{#3}}}}
\newcommand{\pderx}[2][]{\pder[#1]{x^{#2}}}
\newcommand{\pdergx}[2][]{\pderx[g_{#1}]{#2}}
\newcommand{\half}{\frac{1}{2}}
\newcommand{\hfpt}{\hspace{5pt}}
\newcommand{\ddfrac}[2]{\frac{{#1}^2}{{#2}^2}}
\newcommand{\beq}{\begin{equation}}
\newcommand{\beql}[1]{\begin{equation}\label{#1}}
\newcommand{\eeq}{\end{equation}}
\newcommand{\eeqp}{\;\;\;.\end{equation}}
\newcommand{\eeqc}{\;\;\;,\end{equation}}
\newcommand{\GaT}[3]{\Gamma^{#1}_{#2 #3}}
\newcommand{\pderGaTx}[4]{\pderx[\GaT{#1}{#2}{#3}]{#4}}
\newcommand{\Christfinside}[3]{\pdergx[#3 #1]{#2} + \pdergx[#2 #3]{#1} - \pdergx[#1 #2]{#3}}
\newcommand{\Christf}[4]{\Gamma^{#1}_{#2 #3}=\half g^{#1 #4}(\Christfinside{#2}{#3}{#4})}
\newcommand{\Ricchiinside}[2]{\pder[\Gamma^l_{#1 #2}]{x^l} - \pder[\Gamma^l_{#1 l}]{x^{#2}} 
    + \GaT{l}{#1}{#2}\GaT{m}{l}{m} - \GaT{m}{#1}{l}\GaT{l}{#2}{m}}
\date{\today}
\author{山田龍}
\title{113}
\begin{document}
\maketitle
\section{開いた等方モデル}
\beq
ds^2 = c^2 dt^2 - a^2(t)\{d\chi^2 + \sinh^2\chi(d^2\theta + \sin^2\theta\phi^2)\}
\eeq
$cdt = ad\eta$とパラメータを置き換える。
\beq
    ds^2 = a^2(\eta)\{d^2\eta - d\chi^2 - \sinh^2\chi(d^2\theta + \sin^2\theta\phi^2)\}
\eeq
閉じたモデルのときと代数は同じであるので結果は変数の置き換えで形式的に得られる。
$\eta \rightarrow i \eta, \chi \rightarrow i\chi, a \rightarrow ia$と置き換える。
熱力学関係式から、エントロピーを一定とみなして
\beq\label{eq:1133}
    3\ln a = - \int \frac{d\epsilon}{\epsilon + p} + const
\eeq
アインシュタイン方程式の$00$の方程式から、
\beq\label{eq:1134}
\frac{8\pi k}{c^4} \epsilon = \frac{3}{a^4}(a^{\prime 2} - a^2)
\eeq
\eqref{eq:1134}から$\eta$を求める手順を踏めば、
\beq
    \dder[a]{\eta} = \pm a\sqrt{\frac{8\pi k}{3c^4}\epsilon a^2+1}
\eeq
\beq
    \eta = \pm \int \frac{da}{a\sqrt{\frac{8\pi k}{3c^4}\epsilon a^2 + 1}}
\eeq
塵状物体の状態方程式
\beq\label{eq:1135}
    \epsilon = \mu c^2,p = 0
\eeq
を\eqref{eq:1133}に使って、
\beq
\mu a^3 = const = \frac{3c^2}{4\pi k}a_0
\eeq
を得る。\eqref{eq:1135}を計算する。
\begin{align}
    \eta &= \pm \int \frac{da}{a\sqrt{\frac{8\pi k}{3c^4}\epsilon a^2 + 1}}\\
         &= \pm \int \frac{da}{\sqrt{\frac{8\pi k}{3c^4}\epsilon a^4 + a^2}}\\
         &= \pm \int \frac{da}{\sqrt{(a - a_0)^2 - a_0^2}}\\
         &= \cosh^{-1}(\frac{a+a_0}{a_0})
\end{align}
これから、
\beq
    a = a_0(\cosh \eta - 1)
\eeq
$c dt = a d\eta$より、
\beq
    dt = \frac{a_0}{c}(\cosh\eta - 1)d\eta
\eeq
\beq
    t = \frac{a_0}{c}(\sinh \eta - \eta)
\eeq
塵状物質に対しても式を得た。閉じたモデルでは$t \propto \eta - \sin\eta$であった。
$\eta \ll 1$での物質の密度は、
\begin{align}
    a \sim \half a_0  \eta^2\\
    t \sim \frac{a_0}{6c}\eta^3\\
    a \sim (\frac{9a_0 c^2}{2})^{\frac{1}{3}}t^{\frac{2}{3}}\\
    \mu \sim \frac{a_0}{a^3}\frac{3c^2}{4\pi k} = \frac{1}{6\pi kt}
\end{align}
\subsection{}
大きな密度では
\beq
    p = \frac{\epsilon}{3}
\eeq
から\eqref{eq:1133}を計算して、
\beq
    \epsilon a^4 = const = \frac{3c^4a_1^2}{8\pi k}
\eeq
$a,t$は、\eqref{eq:1135}より、
\beq
    a = a_1 \sinh\eta ,\eeq
\beq
t = \frac{a_1}{c}(\cosh\eta - 1)
\eeq
$\eta \ll 1$のもとでは、
\begin{align}
    a \sim a_1 \eta\\
    t \sim \frac{a_1}{c}\frac{\eta^2}{2}\\
    a = \sqrt{2ca_1t}
\end{align}
\subsection{}
曲率半径無限大の場合を考えることができる。
\beq
    ds^2 = c^2 dt^2 - b^2(t)(dx^2 + dy^2 + dz^2)
\eeq
この形に$ds^2$を書けば、b,tの関係を定めるために解くべき式は
\begin{align}
    \frac{8\pi k}{c^2} &= \frac{3}{b^2}(\dder[b]{t})^2\\
    3\ln b &= - \int \frac{d\epsilon}{p + \epsilon} + const
\end{align}
圧力が小さいとして解けば、
\begin{align}
    \mu b^3 = const\\
    3 \ln b = - \int \frac{d\epsilon}{p + \epsilon} + const
\end{align}
tを小さくすれば圧力は無視できなくなるので、$p = \frac{\epsilon}{3}$を考えて、
\begin{align}
    \epsilon b^4 = const\\
    b = const \sqrt{t}
\end{align}
$t=0$に特異点がある。
\end{document}

