\documentclass{jsarticle}
\usepackage{amssymb,amsmath}
\usepackage{amsthm}
\usepackage{newtxtt}
\usepackage[utf8]{inputenc}
\newcommand{\pder}[2][]{\frac{\partial#1}{\partial#2}}
\newcommand{\dder}[2][]{\frac{\mathrm{d}#1}{\mathrm{d}#2}}
\newcommand{\half}{\frac{1}{2}}
\newcommand{\beq}{\begin{equation}}
\newcommand{\beql}[1]{\begin{equation}\label{#1}}
\newcommand{\eeq}{\end{equation}}
\newcommand{\eeqp}{\;\;\;.\end{equation}}
\newcommand{\eeqc}{\;\;\;,\end{equation}}
\newcommand{\xid}{x_i^2}
\newcommand{\lid}{l_i^2}
\newcommand{\aid}{a_i^2}
\newcommand{\sumit}{\sum_{i=1}^3}
\newcommand{\cosdt}{\cos^2\theta}
\newcommand{\sindt}{\sin^2\theta}
\newcommand{\tandt}{\tan^2\theta}
\newcommand{\cosdp}{\cos^2\phi}
\newcommand{\sindp}{\sin^2\phi}

\newtheorem{thm}{定理}
\date{\today}
\author{山田龍}
\title{一様な楕円体のつくる内部のポテンシャル}
\begin{document}
\maketitle
\section{}
楕円体の表面が、
\beq
\sum_{i=1}^3 \frac{\xid}{\aid} = 1
\eeq
で表される楕円体を考える。楕円体内部の点$x_i$をとってこの点から立体角$d\omega$を見込む領域を楕円体から切り取って、切り取られた高さ$R_1, R_2$の2つの三角錐のつくるポテンシャル$d\phi$は、万有引力定数k、密度を$\rho$として
\begin{align}
d\phi &= -k \rho \int_{0}^{R_1} r dr d\omega +
k \rho \int_{0}^{R_2} r dr d\omega\\
      &=  -\half k \rho (R_1^2 + R_2^2) d\omega
\end{align}
これを立体角で積分すれば、求めたいポテンシャルの式は、
\beq
\phi = -\frac{1}{4} k \rho \int_S (R_1^2 + R_2^2) d\omega\label{eq:phyousiki}
\eeq
この積分を実行するためには、準備が必要である。
$l_i = (\sin\theta \cos\phi, \sin\theta \sin\phi ,\cos\theta)$という表記を導入すれば、
$R = R_1, R_2$に対して
\beq
\sum_{i=1}^3 \left (\frac{x_i + R l_i}{a_i} \right) = 1\label{eq:daenl}
\eeq
のようにかける。逆にこの方程式の解が$R_1,R_2$である。
楕円体の表面の方程式を書き直して、
\beq
\sum_{i=1}^3 \frac{r^2 \lid}{\aid} = 1
\eeq
\beq
\sum_{i=1}^3 \frac{\lid}{\aid} = r^2
\eeq
すると$R$についての方程式は、
\beq
\frac{R^2}{r^2} + 2R\sumit \frac{x_i l_i}{a_i} + \sumit \frac{\xid}{\aid} - 1
\eeq
解と係数の関係を考えて、
\beq
R_1^2 + R_2^2 = 4r^4 \left(\sumit \frac{x_i l_i}{a_i} \right) + 2r^2 \left(1 -  \sumit \frac{\xid}{\aid}
\right)
\eeq
\eqref{eq:phyousiki}に代入できて、
\beq
\phi = -\frac{1}{2} k \rho \int_S \left[2r^4 \left(\sumit \frac{x_i l_i}{a_i} \right) + r^2 \left(1 -  \sumit \frac{\xid}{\aid}\right)\right]
\label{eq:steptwo}
\eeq
ここからいくつかの補助定理を証明する。
定理:
\beq
\int _S r^2 d\omega = 2 \pi a_1 a_2 a_3 \int_0^{\infty} \frac{du}{\Delta}
\eeq
$\Delta = \sqrt{(a_1^2 + u)(a_2^2 + u)(a_3^2 + u)}$\\
Proof:
極座標表示では、楕円体の表面の式は
\beq
\frac{1}{r^2} = \frac{\cosdt}{a_3^2} + \sindt(\frac{\cosdp}{a_1^2} + \frac{\sindp}{a_2^2})
\eeq
定理の左辺に代入して、
\begin{align}
    \int_S r^2 d\omega &= \int_0^\pi \int_0^{2\pi}
    \frac{\sin\theta d\theta d\phi}{\frac{\cosdt}{a_3^2} + \sindt(\frac{\cosdp}{a_1^2} + \frac{\sindp}{a_2^2})
}\\
&= 8 \int_0^{\frac{\pi}{2}} \int_0^{\frac{\pi}{2}}
    \frac{\sin\theta d\theta d\phi}{\frac{\cosdt}{a_3^2} + \sindt(\frac{\cosdp}{a_1^2} + \frac{\sindp}{a_2^2})
}\\
&= 8 \int_0^{\frac{\pi}{2}} \int_0^{\infty}
\frac{\sin\theta d\theta dt}{\frac{\cosdt}{a_3^2} +\frac{\sindt}{a_1^2} + t^2(\frac{\cosdt}{a_3^2} + \frac{\sindt}{a_2^2})
}\\
&= 4\pi \int_0^{\frac{\pi}{2}} \frac{\sin\theta d\theta}{\sqrt{\frac{\cosdt}{a_3^2} +\frac{\sindt}{a_1^2}} \sqrt{\frac{\cosdt}{a_3^2} + \frac{\sindt}{a_2^2}
}}\\
&= 4\pi a_1 a_2 a_3^2
\int_0^{\frac{\pi}{2}} \frac{\sec^2\theta \sin\theta d\theta}
{\sqrt{a_3^2 +a_1^2\tandt} \sqrt{a_3^2 + a_2^2 \tandt}}
\end{align}
途中で、
\beq
\int_0^{\infty} \frac{dt}{1+t^2} = \int_0^{\frac{\pi}{2}} d\theta= \frac{\pi}{2}
\eeq
を使った。
ここで、$u = a_3^2 \tandt, du = 2a_3^2 \sin\theta \sec^3\theta d\theta$の置き換えを考えて、
\begin{align}
    \int_S r^2 d\omega &= 2\pi a_1 a_2 \int_0^\infty \frac{\cos\theta du}{\sqrt{a_1^2 + u}\sqrt{a_2^2 + u}}\\
                       &= 2 \pi a_1 a_2 a_3 \int_0^\infty \frac{du}{\Delta}\\
                       &= 2\pi I
\end{align}
次に、
\beq
A_i = a_1 a_2 a_3 \int_0^{\infty} \frac{du}{\Delta (a_i^2 + u)}
\eeq
という記号を定義する。
定理:
\beq
\int_S r^2 l_i^2 = 2\pi a_i^2 A_i
\eeq
Proof:
$i=3$の場合を考える。
\begin{align}
    \int_S r^2 l_3^2 &= \int_S r^2 \cosdt\\
                     &= 2 \pi a_1 a_2 a_3 \int_0^\infty \frac{\cosdt}{\Delta}du\\
                     &= 2\pi a_3^2 A_3\\
\end{align}
対称性から他の場合も示される。適切な置き換えによって同じ証明になる。\\
$I,A$の定義から、
\beq
A_i = \frac{I}{a_i} - \frac{1}{a_i}\pder[I]{a_i}
\eeq
は右辺を実行すれば直ちに得られる。\\
定理:
\beq
\int_S r^4 l_i^2 = \pi a_i^3 \pder[I]{a_i}
\eeq
Proof:
\beq
    \int_s r^2 d\omega = 2\pi I\\
\eeq
両辺微分して、
\beq
\int_S r\pder[r]{a_i} d\omega = \pi \pder[I]{a_i}\label{eq:28}\\
\eeq
一方、
\beq
\frac{1}{r^2} = \sumit \frac{\lid}{\aid}
\eeq
微分して
\beq
\frac{1}{r^3}\pder[r]{a_i} = \sumit \frac{\lid}{a_i^3}
\eeq
この式を、\eqref{eq:28}へ代入すると、定理の式を得る。
ここまでで、\eqref{eq:steptwo}を計算する準備は整った。
\begin{align}
    \phi &= -\frac{1}{2} k \rho \int_S \left[2r^4 \left(\sumit \frac{x_i l_i}{a_i} \right) + r^2 \left(1 -  \sumit \frac{\xid}{\aid}\right)\right]\\
         &= -k \rho \pi \left[\sumit \xid \left( \frac{1}{a_i}\pder[I]{a_i} - \frac{I}{\aid} \right) + I \right]\\
         &= -k \rho \pi \left(
             I - \sumit A_i \xid
         \right)\\
         &= -k \rho \pi a_1 a_2 a_3 \int_0^\infty \left(1 - \sumit \frac{\xid}{\aid + u}\right) \frac{du}{\Delta}
\end{align}

\end{document}

