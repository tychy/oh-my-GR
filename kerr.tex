\documentclass[twocolumn]{jsarticle}
\usepackage{amssymb,amsmath}
\usepackage{newtxtt}
\usepackage[utf8]{inputenc}
\newcommand{\pder}[2][]{\frac{\partial#1}{\partial#2}}
\newcommand{\dder}[2][]{\frac{\mathrm{d}#1}{\mathrm{d}#2}}
\newcommand{\ppder}[2][]{\frac{\partial^2#1}{{\partial#2}^2}}
\newcommand{\pikder}[3][]{\frac{\partial^2#1}{{\partial#2 \partial#3}}}
\newcommand{\pikdergx}[3][]{\frac{\partial^2 g_{#1}}{{\partial x^{#2} \partial x^{#3}}}}
\newcommand{\pderx}[2][]{\pder[#1]{x^{#2}}}
\newcommand{\pdergx}[2][]{\pderx[g_{#1}]{#2}}
\newcommand{\half}{\frac{1}{2}}
\newcommand{\hfpt}{\hspace{5pt}}
\newcommand{\ddfrac}[2]{\frac{{#1}^2}{{#2}^2}}
\newcommand{\beq}{\begin{equation}}
\newcommand{\beql}[1]{\begin{equation}\label{#1}}
\newcommand{\eeq}{\end{equation}}
\newcommand{\eeqp}{\;\;\;.\end{equation}}
\newcommand{\eeqc}{\;\;\;,\end{equation}}
\newcommand{\GaT}[3]{\Gamma^{#1}_{#2 #3}}
\newcommand{\pderGaTx}[4]{\pderx[\GaT{#1}{#2}{#3}]{#4}}
\newcommand{\Christfinside}[3]{\pdergx[#3 #1]{#2} + \pdergx[#2 #3]{#1} - \pdergx[#1 #2]{#3}}
\newcommand{\Christf}[4]{\Gamma^{#1}_{#2 #3}=\half g^{#1 #4}(\Christfinside{#2}{#3}{#4})}
\newcommand{\Ricchiinside}[2]{\pder[\Gamma^l_{#1 #2}]{x^l} - \pder[\Gamma^l_{#1 l}]{x^{#2}}
    + \GaT{l}{#1}{#2}\GaT{m}{l}{m} - \GaT{m}{#1}{l}\GaT{l}{#2}{m}}
\date{\today}
\author{山田龍}
\title{Kerr解の導出}
\begin{document}
\maketitle
\section{一般相対論内山より}
\subsection{アインシュタイン方程式}
計量を
\beq
g_{ij} = \eta_{ij} + al_il_j
\eeq
とかく。$\eta$はミンコフスキー計量。
いま、
\beq\label{eq:etall}
    \eta^{ij} l_il_j = 0
\eeq
を条件として課す。
$l$がヌルベクトルであることを表す。
いま
\beq
l^i = \eta^{ij}l_j
\eeq
が添字を上付きにすると定義しよう。
ヌルベクトルであることが確認される。
\beq
    l^il_j = 0
\eeq
反変計量テンソルは、
\beq
g^{ij} = \eta^{ij} - al^il^j
\eeq
これは
\begin{align}
    g^{ij}g_{jk} &= \delta^i_k\\
    g^{ij} &= \eta^{ij} + bl^il^j\\
    (\eta^{ij} + bl^il^j)(\eta_{jk} + bl_jl_k) &= \delta^i_k + (a + b)l^il_k + abl^ul^jl_jl_k
\end{align}
より$b = -a$がわかることから得た。
\eqref{eq:etall}より空間の計量を使って
\beq
 l^i = g^{ij}l_j 
\eeq
計量からクリストッフェル記号を計算したいので、$l$の微分について考える。
\beq
\Christf{i}{j}{k}{m}
\eeq
から、
\begin{align}
    \GaT{i}{j}{k}l^k &= \frac{g^{im}}{2}(\Christfinside{j}{k}{m})l^k\\
                     &= \frac{ag^{im}}{2}((l_ml_j)_{,k} + (l_kl_m)_{,j} - (l_jl_k)_{,m})l^k\\
                     &= \frac{a}{2}(l^il_j)_{,k}l^k
\end{align}
共変微分が偏微分と一致することを確認する。
書く
行列式が回転に対して不変であることを使って、
あるベクトル$l$を考えて計算すれば回転に対してかわらない
アインシュタイン方程式が得られる。
リーマンテンソル、クリストッフェル記号をaのべきの形で書いて、
アインシュタイン方程式がaによらないことから方程式を4つ得る。
\end{document}

